%% Based on a TeXnicCenter-Template by Tino Weinkauf.
%%%%%%%%%%%%%%%%%%%%%%%%%%%%%%%%%%%%%%%%%%%%%%%%%%%%%%%%%%%%%

%%%%%%%%%%%%%%%%%%%%%%%%%%%%%%%%%%%%%%%%%%%%%%%%%%%%%%%%%%%%%
%% HEADER
%%%%%%%%%%%%%%%%%%%%%%%%%%%%%%%%%%%%%%%%%%%%%%%%%%%%%%%%%%%%%
\documentclass[letterpaper,onside,11pt,review]{report}
% Alternative Options:
%	Paper Size: a4paper / a5paper / b5paper / letterpaper / legalpaper / executivepaper
% Duplex: oneside / twoside
% Base Font Size: 10pt / 11pt / 12pt


%% Packages
\usepackage[paperwidth=17cm, paperheight=22.5cm, bottom=2.5cm, right=2.5cm]{geometry}
\usepackage{algorithm}
\usepackage{amsmath}
\usepackage{amssymb}
\usepackage{amsthm} %paquete para s\'imbolo matem\'aticos
\usepackage[spanish,es-nodecimaldot]{babel} %francais, polish, spanish, ...
\usepackage[utf8]{inputenc}
\usepackage{enumerate}
\usepackage{fancyhdr} %%Fancy headings
\usepackage[T1]{fontenc}
\usepackage{graphicx}
%\usepackage{subfig} %%Subfigures inside a figure
%\usepackage{pst-all} %%PSTricks - not useable with pdfLaTeX
\usepackage[pdftex,
            pdfauthor={CARLOS OMAR PARDO G\'OMEZ},
            pdftitle={T\'iTULO DE LA TESIS},
            pdfsubject={\'aREA DE LA TESIS},
            pdfkeywords={PALABRAS CLAVE},
            pdfproducer={Latex con hyperref},
            pdfcreator={pdflatex}]{hyperref}
\usepackage[ansinew]{inputenc}
\usepackage{lmodern} %Type1-font for non-english texts and characters
\usepackage{longtable} %%For tables, that exceed one page
\usepackage[round]{natbib}
\usepackage{subfig} %para poner subfiguras
\usepackage[nottoc]{tocbibind}
\usepackage{url}
\usepackage{fixltx2e}

\newcommand*{\finalnamedelim}{%
   \ifnumgreater{\value{liststop}}{2}{\finalandcomma}{}%
   \addspace\&\space}

%% Espaciado
\usepackage{setspace}
%\singlespacing        %% 1-spacing (default)
\onehalfspacing       %% 1,5-spacing
%\doublespacing        %% 2-spacing

%%% Norma %%%
%\newcommand{\norm}[1]{\left\lVert#1\right\rVert}

%%% Valor absoluto %%%
\usepackage{mathtools}
\DeclarePairedDelimiter\abs{\lvert}{\rvert}%
\DeclarePairedDelimiter\norm{\lVert}{\rVert}%
% Swap the definition of \abs* and \norm*, so that \abs
% and \norm resizes the size of the brackets, and the 
% starred version does not.
\makeatletter
\let\oldabs\abs
\def\abs{\@ifstar{\oldabs}{\oldabs*}}
%
\let\oldnorm\norm
\def\norm{\@ifstar{\oldnorm}{\oldnorm*}}
\makeatother
%%% %%%

\newtheorem{defin}{Definici\'on}

%%%%%%%%%%%%%%%%%%%%%%%%%%%%%%%%%%%%%%%%%%%%%%%%%%%%%%%%%%%%%
%%  Documento
%%%%%%%%%%%%%%%%%%%%%%%%%%%%%%%%%%%%%%%%%%%%%%%%%%%%%%%%%%%%%
\begin{document}

\pagestyle{empty} %No headings for the first pages.

\title{Tesis} %Con este nombre se guardar\'a el proyecto en writeLaTex

\begin{titlepage}
\begin{center}

\textsc{\Large Instituto Tecnol\'ogico Aut\'onomo de M\'exico}\\[2em]

%Figura
\begin{center}
	\includegraphics{DocumentosLaTex/ITAM_2016}
\end{center}

\vspace{1em}

{\sc \large {\bf UN M\'ETODO BAYESIANO Y NO PARAM\'ETRICO DE REGRESI\'ON SOBRE CUANTILES, MEDIANTE EL USO DE PROCESOS GAUSSIANOS Y PROCESOS DE DIRICHLET}}\\[3em]

\textsc{\large Tesis}\\[1em]

\textsc{que para obtener el t\'itulo}\\[1em]

\textsc{Licenciado en Matem\'aticas Aplicadas}\\[1em]

\textsc{presenta}\\[1em]

\textsc{\Large Carlos Omar Pardo G\'omez}\\[1em]

\textsc{\large Asesor: Dr. Juan Carlos Mart\'inez Ovando}

\end{center}

\vspace*{\fill}
\textsc{M\'exico, D.F. \hspace*{\fill} 2017}

\end{titlepage}


%----------------------------------------------------------------------------------------
%   DECLARACI\'oN
%----------------------------------------------------------------------------------------
\thispagestyle{empty}
\vspace*{\fill}
\begingroup
Con fundamento en los art\'iculos 21 y 27 de la Ley Federal del Derecho de Autor y como titular de los derechos moral y patrimonial de la obra titulada "\textbf{UN M\'ETODO BAYESIANO Y NO PARAM\'ETRICO DE REGRESI\'ON SOBRE CUANTILES, MEDIANTE EL USO DE PROCESOS GAUSSIANOS Y PROCESOS DE DIRICHLET}", otorgo de manera gratuita y permanente al Instituto Tecnol\'ogico Aut\'onomo de M\'exico y a la Biblioteca Ra\'ul Baill\'eres Jr., la autorizaci\'on para que fijen la obra en cualquier medio, incluido el electr\'onico, y la divulguen entre sus usuarios, profesores, estudiantes o terceras personas, sin que pueda percibir por tal divulgaci\'on una contraprestaci\'on.

\centering

\hspace{3em}

\textsc{Carlos Omar Pardo G\'omez}

\vspace{5em}

\rule[1em]{20em}{0.5pt} % L\'inea para la fecha

\textsc{Fecha}
 
\vspace{8em}

\rule[1em]{20em}{0.5pt} % L\'inea para la firma

\textsc{Firma}

\endgroup
\vspace*{\fill}

%----------------------------------------------------------------------------------------
%   DEDICATORIA
%----------------------------------------------------------------------------------------
\pagestyle{empty}
\frontmatter

\chapter*{}
\begin{flushright}
\textit{A Mago.}
\end{flushright}


%----------------------------------------------------------------------------------------
%   AGRADECIMIENTOS
%----------------------------------------------------------------------------------------
\chapter*{Agradecimientos}
%\markboth{AGRADECIMIENTOS23}{AGRADECIMIENTOS} % encabezado 
¡Muchas gracias a todos!


%----------------------------------------------------------------------------------------
%   PREFACIO
%----------------------------------------------------------------------------------------
\pagestyle{plain}
\chapter*{Prefacio}
\markboth{PREFACIO23}{PREFACIO} % encabezado 

%%  Contenido
\addtocontents{toc}{\protect\vspace*{\baselineskip}}

El centro de esta tesis es describir un modelo de \textit{regresi\'on sobre cuantiles}, debido a las bondades que tiene sobre el com\'unmente usado an\'alisis de \textit{regresi\'on sobre la media}. Adem\'as, aceptando los axiomas de la Estad\'istica Bayesiana, permite incorporar conocimiento previo del modelador. Por otra parte, el modelo es no param\'etrico, aumentando la flexibilidad en su forma.

El cap\'itulo 1 introduce la importancia de las aproximaciones disintas a la \textit{regresi\'on sobre la media}, así como la evoluci\'on hist\'orica de este tipo de modelos. El cap\'itulo 2 introduce al Paradigma bayesiano y sus m\'etodos.
El cap\'itulo 3 se centra en los Procesos Gaussianos, como herramienta para describir la relaci\'on entre una variable dependiente y un conjunto de variables independientes. El cap\'itulo 4 describe a los Procesos de Dirichlet, una distribuci\'on de distribuciones no param\'etricas, que permiten modelar el error aleatorio. El cap\'itulo 5 explora modelos bayesianos para la \textit{regresi\'on sobre cuantiles}, hasta llegar al modelo central de esta tesis, y su respectiva implementaci\'on computacional. El cap\'itulo 6 describe aplicaciones del modelo, as\'i como los resultados obtenidos de evaluarlo en diversos conjuntos de datos. Finalmente, el cap\'itulo 7 hace referencia a las conclusiones finales de esta tesis, adem\'as de describir el trabajo futuro que se podr\'ia desarrollar, retomando las ideas de \'esta.

%----------------------------------------------------------------------------------------
%   ÍNDICE
%----------------------------------------------------------------------------------------

\tableofcontents

%%%%%%%%%%%%%%%%%%%%%%%%%%%%%%%%%%%%%%%%%%%%%%%%%%%%%%%%%%%%%
%%    Capitulos
%%%%%%%%%%%%%%%%%%%%%%%%%%%%%%%%%%%%%%%%%%%%%%%%%%%%%%%%%%%%%

\chapter{Introducci\'on}

Detr\'as de cualquier modelo de regresi\'on, la intenci\'on es explicar una variable dependiente en funci\'on de un conjunto de variables independientes, suponiendo cierto error aleatorio. Ha sido com\'un resumir esta dependencia mediante alguna medida de tendencia central, condicionada a los valores de las covariables.

La medida de tendencia central tradicionalmente usada ha sido la media, dando lugar a los modelos de \textit{regresi\'on sobre la media}, con sus variantes lineal o no lineal, simple o m\'ultiple. Este tipo de modelos tiene un buen n\'umero de ventajas, entre las que destacan el bajo costo de calcularlos y la facilidad de interpretaci\'on. Sin embargo, como mencionan \cite{Hao_FrecQuantReg}, tienen tres grandes limitaciones.

La primera es que al resumir la relaci\'on entre la variable dependiente y las independientes con el valor esperado, no necesariamente se puede extender la inferencia a valores lejanos a la media, que suelen ser de inter\'es en ciertos contextos, como los seguros o las finanzas.

La segunda es que los supuestos de este tipo de modelos no siempre se cumplen en el mundo real. Por ejemplo, el supuesto de homocedasticidad; es decir, la varianza no es constante, sino cambia en sincron\'ia con distintos valores de las covariables. Tambi\'en es posible que fen\'omenos de estudio tengan distribuciones de colas pesadas, principalmente en las ciencias sociales. Esto da lugar a valores at\'ipicos, mismos que no suelen ser manejados como se desear\'ia por los modelos de \textit{regresi\'on sobre la media}.

La tercera es que no permiten conocer las propiedades y forma de la distribuci\'on completa. Por ejemplo, la asimter\'ia es una caracter\'istica importante en estudios de ingreso, impuestos, esperanza de vida y, en general, en estudios de desigualdad.

Debido a esto, desde mitades del siglo XVIII han surgido alternativas a este tipo de modelos, siendo la primera los modelos de \textit{regresi\'on sobre la mediana}. De nueva cuenta se busc\'o una medida de tendencia central, pero con otras bondades. Por ejemplo, ser una mejor medida informativa para distribuciones asim\'etricas y menos susceptible a valores at\'ipicos. 

As\'i como los \textit{modelos de regresi\'on sobre la media} son com\'unmente relacionados con la minimizaci\'on de los errores cuadr\'aticos, los \textit{modelos de regresi\'on sobre la mediana} lo son con la minimizaci\'on de los errores absolutos. Debido a la no diferenciabilidad, tuvieron que pasar muchos años para que lograran ser viables, hasta que el poder computacional y los algoritmos de Programaci\'on Lineal lo permitieron.

Cabe recordar que el cuantil $p$-\'esimo es aquel valor tal que una proporci\'on $p$ de los valores est\'an por debajo de \'el, y una proporci\'on $1-p$, por arriba. As\'i, la mediana es un caso particular de un cuantil, espec\'ificamente el $0.5$-\'esimo. Esto abre la idea de que otros cuantiles tambi\'en podr\'ian ser modelados en funci\'on de las covariables, y no necesariamente tienen que ser una medida de tendencia central. 

Los \textit{modelos de regresi\'on sobre cuantiles} fueron introducidos por \cite{Koenker_QuantReg}, y han permitido concentrarse en valores de inter\'es para los modeladores, sin importar que est\'en alejados de la media. Adem\'as, el c\'alculo de diversos cuantiles para un mismo fen\'omeno ha permitido entender mejor la forma y propiedades de las distribuciones condicionales de la variable de respuesta.

En el paradigma bayesiano, el desarrollo de este tipo de modelos ha sido lento. \cite{Walker_BayesAccFail}, \cite{Kottas_BaySemiparamMed} y \cite{Hanson_PolyaTrees} desarrollaron modelos para la mediana, suponiendo una distribuci\'on no param\'etrica del error. \cite{Yu_BayQuantReg} y \cite{Tsionas_BayQuantInf} desarrollaron inferencia param\'etrica, basados en la distribuci\'on asim\'etrica de Laplace para los errores. Por otro lado, \cite{Lavine_LikeQuant} y \cite{Dunson_ApproxBayes} usaron un perspectiva distinta y propusieron una aproximaci\'on de la verosimilitud para cuantiles.

Las limitantes de estos trabajos han sido que, aunque han dado formas flexibles a la distribuci\'on del error, han estado basados en funciones lineales para describir la relaci\'on entre la variable de respuesta y las covariables, o han tenido que recurrir a estimaciones no probabil\'isticas o no bayesianas, para resolver alguna parte del problema.

Entendiendo como \textit{modelo no par\'ametrico} a aquel en el que el n\'umero de par\'ametros no est\'a previamente definido, sino que depende de los datos, esta tesis rescata las ideas de \cite{Kottas_NotParamQuantReg} para proponer un modelo bayesiano y no param\'etrico, \'util en el contexto de \textit{regresi\'on sobre cuantiles}.

\newpage

\chapter[Paradigma bayesiano]{Paradigma bayesiano\raisebox{.3\baselineskip}{\normalsize\footnotemark}}\footnotetext{Las ideas de este cap\'itulo son retomadas de \cite{Denison_BayesMethods} y \cite{Bannerjee_BayLinMod}.}



\section{Axiomas}
Esta tesis da como aceptados los axiomas de la Estad\'istica Bayesiana, detallados durante muchos años en la literatura. Por ejemplo, pueden ser encontrados en \cite{Fishburn_Axioms}. Por lo tanto, entiende a dicho paradigma como el coherente para hacer estad\'istica, cuando una toma de decisi\'on con incertidumbre es el objetivo final del estudio. 

\section{Inferencia}

Un problema clásico de la estad\'istica es el de hacer predicci\'on, utilizando la informaci\'on de los datos que ya han sido observados. Por ejemplo, es posible pensar que ya se tiene el conjunto de $n$ datos observados $\{y_1, ..., y_n\}$ y se desea hacer predicci\'on acerca del valor del dato $y_{n+1}$, que a\'un no ha sido observado. Para esto, se podr\'ia usar la probabilidad condicional
\begin{equation*}
    p(y_{n+1}|y_1,...,y_n) =
    \frac{p(y_{n+1} \cap \{y_1, ..., y_n\})}{p(y_1, ..., y_n)} =
    \frac{p(y_1, ..., y_n,y_{n+1})}{p(y_1, ..., y_n)},
\end{equation*}
pero esto requerir\'ia conocer la funci\'on conjunta, misma que puede ser compleja por la estructura de dependencia de los datos.

No tiene mucho sentido suponer una estructura de independencia entre ellos, porque entonces el conjunto de observaciones $\{y_1, ..., y_n\}$ no dar\'ia informaci\'on alguna para $y_{n+1}$. Pero se puede suponer una distribuci\'on condicionalmente independiente. Es decir, se supone que cada una de las $y_i$'s tiene una misma distribuci\'on param\'etrica, con vector de par\'ametros $\theta$, y se cumple que
\begin{equation*}
    p(y_{k+1},y_k | \theta) =  p(y_k | \theta) \times p(y_{k+1} | \theta).
\end{equation*}

Siguiendo el mismo razonamiento, es posible obtener que
\begin{equation*}
    p(y_1,...,y_n | \theta) = \prod_{i=1}^n p(y_i|\theta). 
\end{equation*}

Dado que se desea hacer inferencia, y al igual que en otros paradigmas, se supone a $\theta$ como constante, pero desconocido. Una particularidad del paradigma bayesiano es expresar la incertidumbre que tiene el modelador acerca del valor verdadero mediante la asignaci\'on de una distribuci\'on a $\theta$, sujeta la informaci\'on inicial o conocimiento previo que se tenga del fen\'omeno $(H)$. Es decir, $p(\theta|H)$. Como una simplificaci\'on de la notaci\'on, en la literatura normalmente se escribe como $p(\theta) = p(\theta|H)$ y se conoce como la \textit{probabilidad inicial} del par\'ametro.

Regresando al problema inicial, y bajo los supuestos reci\'en mencionados, es importante notar que es posible escribir
\begin{equation*}
    p(y_{n+1} | y_1,...,y_n) =
    \int_{\Theta} 
    p(y_{n+1}|\theta) p(\theta|y_1,...y_n)d\theta,
\end{equation*}
donde a su vez, usando el \textbf{Teorema de Bayes}, se obtiene que
\begin{equation*}
\begin{aligned}
    p(\theta|y_1,...y_n) &=
    \frac{p(y_1,...y_n|\theta)p(\theta)}{p(y_1,...y_n)}\\ &=
    \frac{\left[\prod_{i=1}^n p(y_i|\theta)\right]p(\theta)}
    {\int_{\Theta}\left[\prod_{i=1}^n p(y_i|\theta)\right] p(\theta) d\theta},
\end{aligned}
\end{equation*}
que en el paradigma bayesiano se conoce como la \textit{probabilidad posterior} del par\'ametro.

Cabe observar que el denominador no depende de $\theta$, por lo que normalmente la probabilidad no se expresa como una igualdad, sino con la proporcionalidad
\begin{equation*}
    p(\theta|y_1,...,y_n) \propto p(y_1,...,y_n|\theta)p(\theta),
\end{equation*}
o en general,
\begin{equation*}
    Posterior \propto Verosimilitud \times Inicial.
\end{equation*}

Es importante notar que bajo este enfoque se obtiene una distribuci\'on completa para  el pron\'ostico de $y_{n+1}$. Esta se puede utilizar para el c\'alculo de estimaciones puntuales o intervalos, que en el caso del paradigma bayesiano son llamados de \textit{probabilidad}, mediante el uso de funciones de utilidad o p\'erdida, que son estudiadas con m\'as detalle en la Teor\'ia de Decisi\'on.

\section{Regresión lineal}

Se piensa un modelo de \textit{regresión a la media} tradicional, donde $y \in \mathbb{R}$ es la variable de respuesta y $x \in \mathbb{R}^n$ es el vector de covariables. La variable $\varepsilon \in \mathbb{R}$ representa el error aleatorio y se distribuye $\varepsilon \sim \mathcal{N}(0,\sigma^2)$, independiente respecto a $x$. De esta manera, entonces se tiene que:
\begin{equation*}
    y = \beta^Tx + \varepsilon \sim \mathcal{N}(\beta^T x,\sigma^2),
\end{equation*}
donde $\beta \in \mathbb{R}^n$ y $\sigma^2 \in \mathbb{R}^+$ se piensan con valores constantes, pero desconocidos, y la tarea es estimarlos.

Para hacer esto, el enfoque bayesiano le asigna una distribución de probabilidad a ambos par\'ametros, reflejando la incertidumbre que tiene el modelador acerca de su valor real. Es decir, sea $H$ la hip\'otesis o el conocimiento previo al que tiene acceso el modelador, se tiene que 
\begin{equation*}
    (\beta,\sigma^2) \sim P(\beta,\sigma^2|H).
\end{equation*}

A partir de este momento se omitir\'a escribir la distribuci\'on condicional respecto a $H$ por simplificaci\'on de la notaci\'on, pero es importante no olvidar su existencia.

Sea $\{(x_i,y_i)| x_i \in \mathbb{R}^n, y_i \in \mathbb{R}, i \in \{1,...,m\} \}$ el conjunto de datos observados, condicionalmente independientes e id\'enticamente distribuidos, de las variables de respuesta y de las covariables. Es posible representar este mismo conjunto con la notaci\'on matricial $\{X,Y | X \in \mathbb{R}^{m \times n}, Y \in \mathbb{R}^m\}$. Sea $\mathcal{E} \in \mathbb{R}^m$ el vector de errores aleatorios, tal que $\mathcal{E} \sim \mathcal{N}(0,\sigma^2 I)$. El modelo se puede reescribir como:
\begin{equation*}
    Y = X\beta + \mathcal{E} \sim \mathcal{N}(X\beta,\sigma^2 I).
\end{equation*}

Por el Teorema de Bayes,
\begin{equation*}
\begin{aligned}
    P(\beta,\sigma^2 | Y, X) 
    &= \frac{P(Y| X, \beta, \sigma^2) \times P(\beta, \sigma^2 | X)}{P(Y | X)} \\
    &= \frac{P(Y| X, \beta, \sigma^2) \times P(\beta, \sigma^2)}{P(Y | X)} \\
    &\propto P(Y| X, \beta, \sigma^2) \times P(\beta, \sigma^2), \\
\end{aligned}
\end{equation*}
donde $P(Y| X, \beta, \sigma^2)$ es la verosimilitud de los datos observados y se puede calcular como $P(Y| X, \beta, \sigma^2) = \mathcal{N}(X\beta,\sigma^2 I) = \prod_{i=1}^m \mathcal{N}(x_i^T\beta,\sigma^2)$. Por otro lado, $P(\beta,\sigma^2) = P(\beta,\sigma^2|H)$ es la distribuci\'on inicial de los par\'ametros.

Por conveniencia anl\'itica, hay una distribuci\'on inicial com\'unmente usada para los par\'ametros $\beta$ y $\sigma$ debido a que es conjugada respecto a la distribuci\'on Normal de los datos. Su nombre es \textit{Normal-Gamma Inversa (NGI)} y se dice que $\beta,\sigma^2 \sim \mathcal{NGI}(M,V,a,b)$, si
\begin{equation*}
\begin{aligned}
    P(\beta,\sigma^2) 
    &= P(\beta|\sigma^2) \times P(\sigma^2) \\
    &= \mathcal{N}(\beta|M, \sigma^2 V) \times \mathcal{GI}(\sigma ^2|a,b) \\
    &= \frac{1}{((2\pi)^n|\sigma^2 V|)^\frac{1}{2}}
       \exp\left(-\frac{1}{2}(\beta-M)^T{(\sigma^2 V)}^{-1}(\beta-M)\right) \\
    &  \text{ }\text{ }\text{ } \times
       \frac{b^a}{\Gamma(a)}(\sigma^2)^{-(a+1)}\exp\left(-\frac{b}{\sigma^2}\right) \\
    &= \frac{b^a}{(2\pi)^\frac{n}{2}{|V|}^\frac{1}{2}\Gamma(a)}(\sigma^2)^{-(a+    (n/2)+1)} \\
    & \text{ }\text{ }\text{ } \times
      \exp\left(-\frac{(\beta-M)^TV^{-1}(\beta-M) + 2b}{2\sigma^2}\right) \\
    &\propto (\sigma^2)^{-(a+(n/2)+1)} \exp\left(-\frac{(\beta-M)^TV^{-1}(\beta-M) + 2b}{2\sigma^2}\right),
\end{aligned}
\end{equation*}
donde $M$ es la media inicial de los coeficientes, $\sigma^2 V$ su varianza, y $a$ y $b$ son los par\'ametros iniciales de forma y medida de $\sigma ^2$. 

Aprovechando la propiedad conjugada, es posible escribir la probabilidad posterior de los par\'ametros como:
\begin{equation*}
\begin{aligned}
    P(\beta,\sigma^2 | Y, X) 
    &\propto P(Y| X, \beta, \sigma^2) \times P(\beta, \sigma^2), \\
    &\propto (\sigma^2)^{-(\bar{a}+(n/2)+1)} \exp\left(-\frac{(\beta-\bar{M})^T\bar{V}^{-1}(\beta-\bar{M}) + 2\bar{b}}{2\sigma^2}\right),
\end{aligned}
\end{equation*}
donde
\begin{equation*}
\begin{aligned}
    \bar{M} &= (V^{-1} + X^TX)^{-1} (V^{-1}M + X^TY), \\
    \bar{V} &= (V^{-1} + X^TX)^{-1}, \\
    \bar{a} &= a + n/2, \\
    \bar{b} &= b + \frac{\bar{M}^TV^{-1}M + Y^TY - \bar{M}^T\bar{V}^{-1}\bar{M}}{2}.
\end{aligned}
\end{equation*}

Es decir, la distribuci\'on posterior de $(\beta,\sigma^2)$ es \textit{Normal - Gamma Inversa}, con par\'ametros $\mathcal{NGI}(\bar{M},\bar{V},\bar{a},\bar{b})$.

Si se tiene una nueva matriz de covariables $X_*$ y se desea hacer predicci\'on de las respectivas variables de salida $Y_*$, es posible hacer inferencia con los datos observados de la siguiente manera:
\begin{equation*}
\begin{aligned}
    P(Y_*|X_*,Y,X)
    &= \int \int P(Y_*|X_*,\beta,\sigma^2) \times P(\beta,\sigma^2|Y,X) d\sigma^2 d\beta \\
    &= \int \int \mathcal{N}(X_*\beta,\sigma^2I) \times P(\beta,\sigma^2|Y,X) d\sigma^2 d\beta.
\end{aligned}
\end{equation*}

Particularmente, si se contin\'ua con el modelo conjugado \textit{Normal - Gamma Inversa / Normal}, es posible encontrar la soluci\'on anal\'itica:
\begin{equation*}
\begin{aligned}
    P(Y_*|X_*,Y,X)
    &= \int \int \mathcal{N}(X_*\beta,\sigma^2I) \times P(\beta,\sigma^2|Y,X) d\sigma^2 d\beta \\
    &= \int \int \mathcal{N}(X_*\beta,\sigma^2I) \times \mathcal{NGI}(\bar{M},\bar{V},\bar{a},\bar{b}) d\sigma^2 d\beta \\
    &= MVSt_{2\bar{a}} 
       \left(
        X_*\bar{M},\frac{\bar{b}}{\bar{a}}\left(I + X_*\bar{V}X_*^T\right)
       \right),
\end{aligned}
\end{equation*}
donde $MVSt$ es la distribuci\'on \textit{t-Student} multivariada, y cuya definici\'on se describe a continuaci\'on. 

\begin{defin}
    Sea $X \in \mathbb{R}^p$ un vector aleatorio, con media, mediana y moda $\mu$, matriz de covarianzas $\Sigma $, y $\nu$ grados de libertad, entonces $X \sim MVSt_{\nu}(\mu,\Sigma)$ si y s\'olo si su funci\'on de densidad es:
\begin{equation*}
    f(x|\mu,\sigma,\nu) = 
    \frac{\Gamma((\nu+p)/2)}{\Gamma(\nu/2)\nu^{p/2}\pi^{p/2}|\Sigma|^{1/2}}
    \left[1 + \frac{1}{\nu} (x-\mu)^T\Sigma^{-1}(x-\mu)\right]^{-\frac{\nu+p}{2}}.
\end{equation*}
\end{defin}

\newpage

\chapter[Procesos Gaussianos]{Procesos Gaussianos}

\section{Motivaci\'on}

\subsection[Modelos de regresi\'on no lineal]{
    Modelos de regresi\'on no lineal
}

En el cap\'itulo anterior se analiz\'o un modelo robusto para realizar regresi\'on hacia una variable de respuesta, dado un cierto conjunto de covariables. Si bien es un modelo con muchas ventajas, es relevante no olvidar que cuenta con un supuesto fuerte: la relación entre la variable dependiente $y$ y las variables independientes $x$ \'unicamente se da de forma lineal. Pero las funciones lineales s\'olo son un pequeño subconjunto del conjunto infinito no-numerable de funciones existentes. Por ello, valdr\'ia la pena analizar si es posible relajar este supuesto y tener un modelo m\'as general.

Una idea inicial para darle la vuelta a este supuesto es redefinir variables, de tal manera que se pueda obtener un polinomio. Por ejemplo, pensemos que $\hat{x}$ es un buen predictor de $y$, pero como polinomio de orden 3, es decir:
\begin{equation*}
    y = \beta_0 + \beta_1\hat{x} + \beta_2\hat{x}^2 + \beta_3\hat{x}^3 + \varepsilon.
\end{equation*}

Entonces, se puede definir el vector $x$ de covariables como $x = (1,\hat{x},\hat{x}^2,\hat{x}^3)$ y aplicar las t\'ecnicas de regresi\'on lineal ya mencionadas.

Otra cr\'itica que se le podr\'ia hacer a este modelo es la rigidez en la interacci\'on entre variables. Para ejemplificar esto, se podr\'ia pensar en un modelo de la forma:
\begin{equation*}
    y = \beta_0 + \beta_1\hat{x}_1 + \beta_2\hat{x}_2 + \beta_3\hat{x_1}\hat{x_2} + \varepsilon.
\end{equation*}

Es posible entonces declarar el vector $x$ de variables de entrada de la forma $x = (1,\hat{x}_1,\hat{x}_2,\hat{x}_1\hat{x}_2)$, y el procedimiento ser\'ia an\'alogo.

Y a\'un es posible dar un siguiente paso, saliendo del terreno de los polinomios y entrando en el de las funciones biyectivas. Se podr\'ia pensar en un caso como el siguiente (donde siempre se cumpla que $\hat{y} > 1$):
\begin{equation*}
\begin{aligned}
    ln(\hat{y}) &= \hat{\beta_0}\hat{x}_1^{\beta_1}\hat{x}_2^{\beta_2} e^{\varepsilon} \\
    \implies ln(ln(\hat{y})) &= ln(\hat{\beta_0}) + \beta_1 ln(\hat{x}_1) + \beta_2 ln(\hat{x}_2) + \varepsilon \\
    \implies y &= \beta_0 + \beta_1 x_1 + \beta_2 x_2 + \varepsilon, 
\end{aligned}
\end{equation*}
donde
\begin{equation*}
\begin{aligned}
    y &= ln(ln(\hat{y})), \\
    \beta_0 &= ln(\hat{\beta_0}), \\
    x_1 &= ln(\hat{x}_1), \\
    x_2 &= ln(\hat{x}_2),
\end{aligned}
\end{equation*}
y el procedimiento se convierte en el ya conocido.

Si bien estos ejemplos permiten ampliar el conjunto de funciones que es posible cubrir usando un modelo de \textit{regresi\'on lineal sobre la media}, permiten darse cuenta de c\'omo se puede complicar la relaci\'on de dependencia entre $y$ y las covariables $x$, de tal manera que muchas funciones pueden no ser descritas con el m\'etodo antes planteado.

As\'i surge la necesidad de buscar un m\'etodo que permita encontrar cualquier tipo relaci\'on entre $y$ y $x$, sin restringirla a un pequeño subconjunto de funciones. El reto es que \'unicamente se tiene tiempo finito para encontrar la mejor estimaci\'on, entre una infinidad no-numerable de opciones.

\subsection[Introducci\'on a los Procesos Gaussianos]{
    Introducci\'on a los Procesos Gaussianos
    \footnote{Las ideas de esta subsecci\'on y de lo que resta del cap\'itulo son inspiradas por \cite{Rasmussen_GauProc}.}
}

Para relajar el supuesto de linearidad, se puede pensar que la relación entre la variable de salida $y$ y las covariables $x$ se da mediante cierta función general $f: \mathbb{R}^n \rightarrow \mathbb{R}$. De esta forma, el modelo se plantea como:
\begin{equation*}
    y = f(x) + \epsilon \sim \mathcal{N}(f(x),\sigma^2).
\end{equation*}

Para continuar con la notación matricial del modelo anterior, sean $Y \in \mathbb{R}^m$ y $X \in \mathbb{R}^{m \times n}$, y $\mathcal{E} \in \mathbb{R}^m$ el vector de errores aleatorios, es posible describir al modelo como
\begin{equation*}
    Y = f(X) + \mathcal{E} \sim \mathcal{N}(f(X),\sigma^2 I),
\end{equation*}
donde
\begin{equation*}
    f(X) =     
    \left[
        \begin{array}{c}
        f(x_1)  \\
        ... \\
        f(x_m)
        \end{array}
    \right], 
    x_i \in \mathbb{R}^n, \forall i \in \{1,...,m\}.
\end{equation*}

Cabe recordar que la función $f$ es pensada constante, pero desconocida. De nueva cuenta, para reflejar la incertidumbre del modelador, es posible darle una distribución de probabilidad. Pero a diferencia del modelo anterior, ya no existe el parámetro $\beta$ al cual canalizarle esta incertidumbre, por lo que ahora tendrá que ser sobre toda la función. Antes de continuar, es útil tener presente la siguiente definición.

\begin{defin}
    Un \textbf{\textit{proceso gaussiano}} ($Y \in \mathbb{R}^m$), es una colección finita de \texit{m}-variables aleatorias que tienen una distribución gaussiana (normal) conjunta.
\end{defin}

Es de utilidad pensar entonces a $f(x)$ como una variable aleatoria, que refleje el desconocimiento del modelador. Particularmente se le puede asignar una distribución Normal, donde la media $m(x)$ y la covarianza $k(x,x')$ reflejen el conocimiento previo que se tenga del fenómeno de estudio. Cabe resaltar que dicha media $m(x)$ y covarianza $k(x,x')$ están en función de $x$, es decir, podrían variar de acuerdo al valor de las covariables. 

Visto de manera matricial y cometiendo un abuso de notaci\'on, dada una matriz de covariables $X \in \mathbb{R}^{m \times n}$, $f(X) \in \mathbb{R}^n$ es un vector aleatorio, que adem\'as depende de variables de entrada, por lo que \textbf{$f(X)$ es un proceso estoc\'astico}. Adem\'as, d\'andole una estructura de covarianza entre los distintos valores de las covariables, $f(X)$ se distribuye Normal Multivariada, donde su vector de medias $M(X)$ y matriz de covarianzas $K(X,X)$ reflejan el conocimiento inicial del modelador.

\newtheorem{obs}{Observaci\'on}

\begin{obs}
    De acuerdo a como se acaba de describir el vector $f(X) \in \mathbb{R}^m$, y tomando en cuenta la Definici\'on 2, además de ser un proceso estoc\'astico, \textbf{$f(X)$ es un proceso gaussiano}.
\end{obs}

% ****************************************************


\section{Inferencia sobre $f$}

\subsection{Definiciones iniciales}

Para las siguientes definiciones se supondrá que $f(x)$ es una variable aleatoria y $f(X)$ un vector aleatorio, con medias y covarianzas conocidas y finitas.

\begin{defin}
Sean $x,x' \in \mathbb{R}^n$. \\

La \textbf{función de medias de f (m\textsubscript{f})} se define como 
\begin{equation*}
    m_f: \mathbb{R}^n \rightarrow \mathbb{R} 
    \mid
    m_f(x) = \mathbb{E}[f(x)].
\end{equation*}

La \textbf{función de covarianzas de f (k\textsubscript{f})} se define como 
\begin{equation*}
    k_f: \mathbb{R}^n \times \mathbb{R}^n \rightarrow \mathbb{R} 
    \mid
    k_f(x, x') = \mathbb{E}[(f(x) - m(x))(f(x') - m(x'))].
\end{equation*}
\end{defin}

\begin{defin}
Sea $X \in \mathbb{R}^m \times \mathbb{R}^n$ y $X' \in \mathbb{R}^p \times \mathbb{R}^n$, es decir,
\begin{equation*}
    X =     
    \left[
        \begin{array}{c}
        x_1  \\
        ... \\
        x_m
        \end{array}
    \right],
    x_i \in \mathbb{R}^n, \forall i \in \{1,...,m\}.
\end{equation*}
\begin{equation*}
    X' =     
    \left[
        \begin{array}{c}
        x_1  \\
        ... \\
        x_p
        \end{array}
    \right],
    x_i \in \mathbb{R}^n, \forall i \in \{1,...,p\}.
\end{equation*}

La \textbf{función matriz de medias de f (M\textsubscript{f})} se define como
\begin{equation*}
    M_f: \mathbb{R}^m \times \mathbb{R}^n: \mathbb{R}^m
    \mid
    M_f(X) =     
    \left[
        \begin{array}{c}
        m_f(x_1)  \\
        ... \\
        m_f(x_m)
        \end{array}
    \right].
\end{equation*}

La \textbf{función matriz de covarianzas de f (K\textsubscript{f})} se define como
\begin{equation*}
    K_f: \mathbb{R}^m \times \mathbb{R}^n: \mathbb{R}^m \times \mathbb{R}^m
    \mid
    K_f(X,X') =     
    \left[
        \begin{array}{ccc}
        k_f(x_1,x_1') & ... & k_f(x_1,x_p')  \\
        ... & ... & ... \\
        k_f(x_m,x_1') & ... & k_f(x_m,x_p')
        \end{array}
    \right].
\end{equation*}
\end{defin}

Dadas estas definiciones, se puede observar que el \textit{proceso gaussiano} $f(X) \in \mathbb{R}^m$ está completamente caracterizado por su función matriz de medias $M_f(X)$ y su función matriz de covarianzas $K_f(X,X')$. Cabe resaltar que si se definen estas funciones de manera general para cualquier $X \in \mathbb{R}^{m \times n}$ que est\'e en el dominio del fen\'omeno a estudiar, en particular estar\'an definidas para cualquier matriz de datos observados o datos a predecir. Por lo tanto, la manera en que se definan estas dos funciones representar\'a el conocimiento inicial que se tiene del objeto de estudio. 

A partir de este punto, y cuando el contexto lo permita, por simplicidad de notaci\'on se omitirá el uso del subíndice $f$ en las funciones reci\'en definidas. Además, cuando se quiera referirse al proceso estoc\'astico $f(X)$ que se distribuye como un \textit{proceso gaussiano}, se har\'a con la siguiente notaci\'on:
\begin{equation*}
    f(X) \sim \mathcal{GP} (M(X),K(X,X)).
\end{equation*}

\subsection{Predicción de observaciones sin ruido}

Sea un conjunto de observaciones sin ruido, es decir, $\{(x_i,f_i)|i=1,...,m \}$, con $f_i=f(x_i)$. En otras palabras, para toda $x_i$, $y_i=f(x_i)$, sin estar sujeta a un error aleatorio. De forma matricial, se puede escribir como $\{(X,f(X))\}$, con $X \in \mathbb{R}^{m \times n}$ y $f(X) \in \mathbb{R}^{m}$. 

Por otro lado, se tiene un conjunto de covariables $X_* \in \mathbb{R}^{p \times n}$, y se desea predecir $f(X_*)$, suponiendo que sigue la misma función $f$ de los datos observados.

La distribución inicial conjunta de los datos observados $f(X)$ y los datos a predecir $f(X_*)$ es: 
\begin{equation*}
    \left[
        \begin{array}{c}
        f(X)  \\
        f(X_*) 
        \end{array}
    \right]  
    \sim \mathcal{N}  
    \left(
        \left[
            \begin{array}{c} 
            M(X) \\ 
            M(X_*) 
            \end{array}
        \right],
        \left[
            \begin{array}{cc}
            K(X,X) & K(X,X_*)  \\
            K(X_*,X) & K(X_*,X_*) 
            \end{array}
        \right]
    \right) 
\end{equation*}

Es momento oportuno para recordar algunas propiedades de la distribuci\'on Normal condicional. 

\newtheorem{prop}{Propiedad}

\begin{prop}
    Sea $X \in \mathbb{R}^p$ un vector aleatorio que tiene distribuci\'on Normal conjunta y est\'a particionado de la siguiente manera:

    \begin{equation*}
        X = 
        \left[
        \begin{array}{c}
            X_1  \\
            X_2
        \end{array}
        \right], 
        \text{ con dimensiones }
            \left[
        \begin{array}{c}
            (p-q)  \\
            q
        \end{array}
        \right],
    \end{equation*}
    
    Entonces, la media $\mu \in \mathbb{R}^p$ y varianza $\Sigma \in \mathbb{R}^{p \times p}$ de $X$ se pueden escribir
    \begin{equation*}
    \begin{aligned}
        \mu &= 
        \left[
        \begin{array}{c}
            \mu_1  \\
            \mu_2
        \end{array}
        \right], 
        \text{ dimensiones }
            \left[
        \begin{array}{c}
            (p-q)  \\
            q
        \end{array}
        \right], \\
        \Sigma &= 
        \left[
        \begin{array}{cc}
            \Sigma_{11} & \Sigma_{12}  \\
            \Sigma_{21} & \Sigma_{22}
        \end{array}
        \right], 
        \text{ dimensiones }
            \left[
        \begin{array}{cc}
            (p-q) \times (p-q)  & (p-q) \times q  \\
            q \times (p-q) & q \times q
        \end{array}
        \right].
    \end{aligned}
    \end{equation*}
    
    La distribuci\'on condicional de $X_2$, sujeta a que $X_1 = a$ es Normal con $X_2|X_1=a \sim \mathcal{N}(X_2|\bar{\mu},\bar{\Sigma})$, con
    
    \begin{equation*}
    \begin{aligned}
        \bar{\mu} &= \mu_2 + \Sigma_{2,1}\Sigma_{11}^{-1}(a-\mu_1) \\
        \bar{\Sigma} &= \Sigma_{22} - \Sigma_{21}\Sigma_{11}^{-1}\Sigma_{12}
    \end{aligned}
    \end{equation*}
\end{prop}

De regreso al modelo, tomando en cuenta que existen datos conocidos, es posible condicionar la distribución conjunta, dadas esas observaciones. Utilizando las propiedades de la distribución Normal condicional, se obtiene que:
\begin{equation*}
    f(X_*)|f(X) 
    \sim \mathcal{N}
    (\bar{M}(X,X_*),\bar{K}(X,X_*)),
\end{equation*}
con
\begin{equation*}
\begin{aligned}
    \bar{M}(X,X_*) &= M(X_*) + K(X_*,X)K(X,X)^{-1}(f(X) - M(X)), \\
    \bar{K}(X,X_*) &= K(X_*,X_*) - K(X_*,X)K(X,X)^{-1}K(X,X_*).
\end{aligned}
\end{equation*}

\begin{obs}
    $f(X_*)|f(X)$ es una colección finita de p-variables aleatorias que tienen una distribuci\'on Normal conjunta, por lo tanto, $\mathbf{f(X_*)|f(X)}$ \textbf{es un proceso gaussiano}. 
\end{obs}

Para confirmar que no existe ruido en las observaciones, es posible sustituir $X_* = X$ y ver los posible valores para $f(X_*)$.
\begin{equation*}
\begin{aligned}
    \mathbb{E}[f(X_*)|f(X)] 
    &= \bar{M}(X,X_*) \\
    &= \bar{M}(X,X) \\
    &= M(X) + K(X,X)K(X,X)^{-1}(f(X) - M(X)) \\
    &= M(X) + f(X) - M(X) \\
    &= f(X) ,
\end{aligned}
\end{equation*}
\begin{equation*}
\begin{aligned}
    Var(f(X_*)|f(X)) 
    &= \bar{K}(X,X_*) \\
    &= \bar{K}(X,X) \\
    &= K(X,X) - K(X,X)K(X,X)^{-1}K(X,X) \\
    &= K(X,X) - K(X,X) \\
    &= 0.
\end{aligned}
\end{equation*}

Es decir, si $X_* = X$, $f(X_*) \sim \mathcal{N}(f(X),0)$. En otras palabras, la media es el vector de valores ya obtenidos $f(X)$ y varianza 0, por lo que se cumple que para cualquier $X$, $f(X)$ tendría siempre un único valor.

\subsection{Predicción de observaciones con ruido}

Ahora se supone un conjunto de observaciones con ruido, es decir, $\{(x_i,y_i)|i=1,...,m \}$, con $y_i=f(x_i) + \epsilon$, donde $\epsilon \sim \mathcal{N}(0,\sigma^2)$ y $\sigma^2 > 0$. Con notación matricial se puede describir a este conjunto como $\{(X,Y)\}$, con $X \in \mathbb{R}^{m \times n}$ y $Y \in \mathbb{R}^{m}$, y el vector de errores aleatorios es $\mathcal{E} \sim \mathcal{N}(0,\sigma^2I)$. Así, se tiene que
\begin{equation*}
    Y = f(X) + \mathcal{E}.
\end{equation*}

Es posible observar que al ser suma de dos Normales, la distribuci\'on inicial de $Y$ ser\'a Normal. Por lo tanto, 
\begin{equation*}
    Y = f(X) + \mathcal{E} \sim \mathcal{N}(M(X), K(X,X) + \sigma^2I).
\end{equation*}

\begin{obs}
    Y es una colección finita de m-variables aleatorias que tienen una distribuci\'on Normal conjunta, por lo tanto, \textbf{$Y$ es un proceso gaussiano}.
\end{obs}

A partir de este punto, en esta sección se supondrá a \textbf{$\sigma^2$ como constante y conocida}, y la atención principal estará sobre la función $f$.

Ahora se piensa en un conjunto de covariables $X_* \in \mathbb{R}^{p \times n}$, y se busca predecir $f(X_*)$, suponiendo que sigue la misma función $f$ de los datos observados. La distribución inicial conjunta de los datos observados $Y$ y los datos a predecir $f(X_*)$ es: 
\begin{equation*}
    \left[
        \begin{array}{c}
        Y \\
        f(X_*)
        \end{array}
    \right]  
    \sim \mathcal{N}  
    \left(
        \left[
            \begin{array}{c} 
            M(X) \\ 
            M(X_*) 
            \end{array}
        \right],
        \left[
            \begin{array}{cc}
            K(X,X) + \sigma^2 I_{m} & K(X,X_*)  \\
            K(X_*,X) & K(X_*,X_*)
            \end{array}
        \right]
    \right), 
\end{equation*}
donde $I_m$ es la matrz identidad de dimensi\'on $m$.

Considerando que ya se cuenta con datos conocidos, se toma la distribuci\'on condicional de la Normal y se obtiene que:
\begin{equation*}
    f(X_*)|Y
    \sim \mathcal{N}
    (\bar{M}(X,X_*,\sigma^2),\bar{K}(X,X_*,\sigma^2)),
\end{equation*}
con
\begin{equation*}
\begin{aligned}
    \bar{M}(X,X_*,\sigma^2) &= M(X_*) + K(X_*,X)(K(X,X) + \sigma^2I_m)^{-1}(Y - M(X)) \\
    \bar{K}(X,X_*,\sigma^2) &= K(X_*,X_*) - K(X_*,X)(K(X,X) + \sigma^2I_m)^{-1}K(X,X_*).
\end{aligned}
\end{equation*}

\begin{obs}
    $f(X_*)|Y$ es una colección finita de p-variables aleatorias que tienen una distribuci\'on Normal conjunta, por lo tanto,
    $\mathbf{f(X_*)|Y}$ \textbf{ es un proceso gaussiano}. 
\end{obs}

Es posible observar que aunque $X_* = X$, no necesariamente se cumple que $f(X_*)|Y = Y$. En primer lugar, porque $\bar{K}(X,X,\sigma^2) \neq 0$, debido al efecto de $\sigma^2$. En segundo lugar, y de nueva cuenta por causa de $\sigma^2$, $\bar{M}(X,X,\sigma^2) \neq Y$.

\section{Varianza}

\subsection{Funciones de covarianza}

Hasta el momento, no se han descrito las caracter\'isticas de la funci\'on de covarianzas de $f$ $(k_f)$. Se empezar\'a por recordar que la \textbf{función de covarianzas de f (k\textsubscript{f})} se define como 
\begin{equation*}
    k_f: \mathbb{R}^n \times \mathbb{R}^n \rightarrow \mathbb{R} 
    \mid
    k_f(x, x') = \mathbb{E}[(f(x) - m_f(x))(f(x') - m_f(x'))],
\end{equation*}
donde $f(x)$ es una variable aleatoria que se distribuye Normal con media $m_f(x)$. Por lo tanto, se deduce que 
\begin{equation*}
    Cov(f(x),f(x')) = k_f(x,x').
\end{equation*}

Cabe resaltar que $k_f$ no es una \textit{covarianza} en general, ni cumple con todas las propiedades, sino \'unicamente describe la covarianza entre dos vectores aleatorios $f(x)$ y $f(x')$, con la misma $f$, sin la intervenci\'on, por ejemplo, de constantes. Para explicar de mejor manera este punto, se da el siguiente ejemplo:
\begin{equation*}
\begin{aligned}
    Cov(af(x) + f(x'), f(x')) &=
    Cov(af(x), f(x')) + Cov(f(x), f(x'))\\
     &= a \times Cov(f(x), f(x')) +  Cov(f(x'), f(x')) \\
     &= a \times k_f(x,x') + k_f(x',x')
\end{aligned}
\end{equation*}

En este orden de ideas, las propiedades que $k_f(x,x')$ tiene que cumplir son
\begin{equation*}
\begin{aligned}
    k_f(x,x') &= k_f(x',x) \text{ (simetr\'ia),} \\
    k_f(x,x) &= Var(f(x)) \geq 0.
\end{aligned}
\end{equation*}

Si bien es cierto que dadas esas restricciones hay una variedad muy grande de funciones con las que se puede describir $k_f(x,x')$, por practicidad, y tomando en cuenta que es un supuesto sensato para la mayor\'ia de los casos, es com\'un describir a la funci\'on $k_f$ en relaci\'on a la distancia entre $x$ y $x'$, $\norm{x,x'}_p$. Es decir, $k_f(x,x') = k_f(\norm{x,x'}_p)$. A este tipo de funciones de covarianza se les denomina \textbf{estacionarias}.

Adem\'as, esta relaci\'on entre covarianza y distancia suele ser inversa, es decir, entre menor sea la distancia, mayor ser\'a la covarianza, y viceversa. De esta manera, para valores $x \approx x'$, se obtendr\'a que $f(x) \approx f(x')$ en la mayor\'ia de los casos, lo que tiene cierto supuesto impl\'icito de que $f$ es una funci\'on continua.

Un ejemplo de este tipo de funciones son las $\mathbf{\gamma}$\textbf{\textit{-exponencial}}, mismas que se definen de la siguiente manera:
\begin{equation*}
    k(x,x') = 
    k(\norm{x,x'}_\gamma;\gamma,\lambda) = 
    exp\left(-\frac{1}{\gamma}
    \left(\frac{\norm{x,x'}_\gamma}{\lambda}\right)^\gamma
    \right),
\end{equation*}
donde $\lambda$ es un par\'ametro de rango. 

Las de uso m\'as com\'un suelen ser la $1$ y $2$\textit{-exponencial}. Ambas tienen la ventaja de ser continuas, pero la $2$\textit{-exponencial} tiene adem\'as la peculiaridad de ser infinitamente diferenciable y, por lo tanto, es suave.

El siguiente ejemplo de funciones estacionarias es la \textbf{\textit{clase de Matérn}}, descrita como
\begin{equation*}
    k(x,x') = k(\norm{x,x'}_1;\nu,\lambda) = 
    \frac{2^{1-\nu}}{\Gamma(\nu)}
    \left(\frac{\sqrt{2\nu}\norm{x,x'}_1}{\lambda}\right)^\nu
    K_{\nu}
    \left(\frac{\sqrt{2\nu}\norm{x,x'}_1}{\lambda}\right)^\nu,
\end{equation*}
donde $K_{\nu}$ es la funci\'on modificada de Bessel y $\Gamma(.)$ es la funci\'on \textit{gamma}. Los casos m\'as utilizados son
\begin{equation*}
\begin{aligned}
    k\left(\norm{x,x'}_1;\nu = \frac{3}{2},\lambda\right) &= 
    \left(1 + \frac{\sqrt{3}\norm{x,x'}_1}{\lambda}\right)
    exp\left(-\frac{\sqrt{3}\norm{x,x'}_1}{\lambda}\right), \\
    k\left(\norm{x,x'}_1;\nu = \frac{5}{2},\lambda\right) &= 
    \left(1 + \frac{\sqrt{3}\norm{x,x'}_1}{\lambda} + \frac{5\norm{x,x'}_1^2}{3\lambda^2}\right)
    exp\left(-\frac{\sqrt{5}\norm{x,x'}_1}{\lambda}\right). \\
\end{aligned}
\end{equation*}

Otra posible funci\'on de covarianza es la \textbf{\textit{racional cudr\'atica}}, caracterizada como 
\begin{equation*}
    k(x,x') = k(\norm{x,x'}_2;\alpha,\lambda) = 
    \left(1 + \frac{\norm{x,x'}_2^2}{2\alpha \lambda^2}\right)^{-\alpha},
\end{equation*}
con $\alpha,\lambda > 0$.

Existen otro tipo de funciones estacionarias que no guardan una relaci\'on inversa entre distancia y covarianza, sino que a cierta distancia aumenta la covarianza, y esto sucede de forma c\'iclica. En otras palabras, este tipo de funciones capturan un componente \textbf{estacional}, normalmente usado en series de tiempo. De esta manera, y siendo $t$ la covariable del tiempo, es posible pensar en una funci\'on de la forma
\begin{equation*}
    k(x,x',t,t';E) = \bar{k}(x,x') + \delta_{\{(|t'-t| \text{ mod } E) = 0\}},
\end{equation*}
donde $\bar{k}$ es alguna de las funciones estacionarias antes mencionadas, $\delta$ es la \textit{delta de Kroenecker} y $E$ es el periodo de estacionalidad. Por ejemplo, $E = 12$ para una serie mensual.

Si se desea suavizar esta componente de estacionalidad para que no sea \'unicamente puntual, es posible describir la covarianza con una funci\'on como la siguiente:
\begin{equation*}
    k(x,x',t,t';E,\lambda) = 
    \bar{k}(x,x') + 
    exp\left(-\frac{1}{\lambda^2}\frac{E}{\pi}sin^2\left(\frac{\pi}{E}|t'-t|\right)\right).
\end{equation*}

\subsection{Varianza del error aleatorio}

Una vez dicho todo lo anterior, el \'unico pendiente restante es dejar de suponer a $\sigma^2$ (la varianza del error aleatorio $\varepsilon$) como una constante conocida. Como ya se mencion\'o anteriormente, si bien se piensa en ella como constante, es posible reflejar la incertidumbre del modelador respecto al valor verdadero con una distribuci\'on de probabilidad. Es decir, si H son las creencias o la informaci\'on previa con la que se cuenta, entonces
\begin{equation*}
    \sigma^2 \sim P(\sigma^2 | H).
\end{equation*}

Es claro que esta distribuci\'on tiene que tener soporte en alg\'un subconjunto de $\mathbb{R^+}$, por lo que la distribuci\'on Gamma o Gamma Inversa son las com\'unmente utilizadas. Suponiendo que es la primera, el conocimiento de $H$ se tendr\'a que traducir en los par\'ametros $\alpha$ y $\beta$.

As\'i, el modelo de Procesos Gaussianos queda especificado como
\begin{equation*}
\begin{aligned}
    y - f(x) | f(x), \sigma^2 &\sim \mathcal{N}(0,\sigma^2) \\
    f(x) &\sim \mathcal{GP}(m(x),k(x,x)) \\
    \sigma^2 &\sim Gamma(\alpha,\beta).
\end{aligned}
\end{equation*}

\newpage

\chapter[Procesos de Dirichlet]{Procesos de Dirichlet\raisebox{.3\baselineskip}{\normalsize\footnotemark}}\footnotetext{Las ideas de este capítulo son retomadas de \cite{Yee_DirProc}.}

\section{Motivaci\'on}
Un Proceso de Dirichlet, visto de manera general, es una distribuci\'on sobre distribuciones. Es decir, cada realizaci\'on de él es en sí misma una distribuci\'on de probabilidad. Adem\'as, cada una de esas distribuciones ser\'a no param\'etrica, debido a que no ser\'a posible describirla con un n\'umero finito de par\'ametros.

Emplear distribuciones de este tipo permite combatir, por un lado, el \textit{subajuste}, debido a que cualquier distribuci\'on se puede representar de manera no param\'etrica. Por otro lado, combate al \textit{sobreajuste} utilizando un enfoque bayesiano para calcular la probabilidad posterior, dando como distribuci\'on inicial a aquella que se percibe como la m\'as factible.

En el caso particular de esta tesis y de su misi\'on de encontrar un modelo bayesiano y no param\'etrico para la \textit{regresi\'on sobre cuantiles}, los Procesos de Dirichlet ser\'an utilizados para ajustar la distribuci\'on del error aleatorio $\varepsilon$.


\section{Inferencia}

\subsection{Definici\'on formal}

Antes de revisar la definici\'on formal de los Procesos de Dirichlet, es conveniente recordar la definici\'on de la distribuci\'on de Dirichlet.

\begin{defin}
    Se dice que un vector aleatorio $x \in \mathbb{R}^p$ se distribuye de acuerdo a la \textbf{distribuci\'on de Dirichlet}  $\mathbf{(x \sim Dir(\alpha))}$ con vector de par\'ametros $\alpha$, espec\'ificamente,
    \begin{equation*}
        x = 
        \left(\begin{array}{c}
            x_1  \\
            \cdots \\
            x_p
        \end{array}\right),
        \qquad
        \alpha = 
        \left(\begin{array}{c}
            \alpha_1  \\
            \cdots \\
            \alpha_p
        \end{array}\right),
    \end{equation*}
    para los cuales se cumplen las restricciones
    \begin{equation*}
    \begin{aligned}
        x_i > 0, \forall i &\in \{1,...,p\} \\
        \sum_{i=1}^p x_i &= 1 \\
        \alpha_i > 0, \forall i &\in \{1,...,p\},
    \end{aligned}
    \end{equation*}
    si su funci\'on de densidad  es
    \begin{equation*}
        f(x|\alpha) = 
        \frac {1}{\mathrm {B} (\alpha)}
        \prod _{i=1}^{p}x_{i}^{\alpha _{i}-1},
    \end{equation*}
    donde $\mathrm{B}$ es la funci\'on Beta multivariada, y puede ser expresada en t\'erminos de la funci\'on $\Gamma$ como 
    \begin{equation*}
       \mathrm{B}(\alpha)=
       \frac {\prod _{i=1}^{p}\Gamma (\alpha _{i})}
       {\Gamma \left(\sum _{i=1}^{p}\alpha _{i}\right)},
       \qquad 
       \alpha =(\alpha _1,\cdots ,\alpha _p). 
    \end{equation*}
    
    La esperanza y varianza de cada $x_i$ son los siguientes:
    
    \begin{equation*}
    \begin{aligned}
        \mathbb{E}[x_i] &= \frac{\alpha_i}{\sum_{k=1}^p \alpha_k} \\
        Var(x_i) &= \frac
        {\alpha_i \left( \sum_{k=1}^p \alpha_k - \alpha_i \right)}
        {\left( \sum_{k=1}^p \alpha_k \right)^2 (\left( \sum_{k=1}^p \alpha_k + 1 \right)}
    \end{aligned}
    \end{equation*}
    
\end{defin}
Es com\'un que esta distribuci\'on sea usada como la inicial conjugada de la distribuci\'on multinomial, debido a que el vector $x$ tiene las mismas propiedades de una distribuci\'on de probabilidad discreta (elementos positivos y que en conjunto suman 1).

Retomando el tema central, en t\'erminos generales, para que una distribuci\'on de probabilidad $G$ se distribuya de acuerdo a un Proceso de Dirichlet, sus distribuciones marginales tienen que tener una distribuci\'on Dirichlet. A continuaci\'on de enuncia una definici\'on m\'as detallada.

\begin{defin}
    Sean $G$ y $H$ dos distribuciones cuyo soporte es el conjunto $\Theta$ y sea $\alpha \in \mathbb{R}^+$. Entonces, si se toma una partici\'on finita cualquiera $A_1,...,A_r$ del conjunto $\Theta$, el vector $(G(A_1),...,G(A_r))$ es aleatorio, porque $G$ tambi\'en lo es.
    
    Se dice que $G$ se distribuye de acuerdo a un \textbf{Proceso de Dirichlet} $\mathbf{(G \sim DP(\alpha,H))}$, con distribuci\'on media $H$ y par\'ametro de concentraci\'on $\alpha$, si
    \begin{equation*}
        (G(A_1),...,G(A_r)) \sim Dir(\alpha H(A_1),...,\alpha H(A_r)), 
    \end{equation*}
    para cualquier partici\'on finita $A_1,...,A_r$ del conjunto $\Theta$.
\end{defin}

Es momento de analizar el papel que juegan los par\'ametros. Sea $Ai \subset \Theta$, uno de los elementos de la partici\'on anterior, y recordando las propiedades de la distribuci\'on de Dirichlet, entonces
\begin{equation*}
\begin{aligned}
    E[G(A_i)] 
    &= \frac{\alpha H(A_i)}{\sum_{k=1}^p \alpha H(A_k)} \\
    &= H(A_i) \\
\end{aligned}
\end{equation*}

\begin{equation*}
\begin{aligned}
    Var(G(A_i)) 
    &= \frac{\alpha H(A_i)\left(\sum_{k=1}^p(\alpha H(A_k)) - \alpha H(A_i)\right)}
       {\left(\sum_{k=1}^p \alpha H(A_k)\right)^2\left(\sum_{k=1}^p(\alpha H(A_k)) + 1\right)} \\
    &= \frac{\alpha^2 [H(A_i)(1 - H(A_i))]}
       {\alpha^2 (1)^2(\alpha + 1)} \\
    &= \frac{H(A_i)(1 - H(A_i))}
       {\alpha + 1}.
\end{aligned}
\end{equation*}

En este orden de ideas, es posible darse cuenta que la distribuci\'on $H$ representa la \textit{distribuci\'on media} del Proceso de Dirichlet. Por otro lado, el par\'ametro $\alpha$ tiene una relaci\'on inversa con la varianza. As\'i, a una mayor $\alpha$, corresponde una menor varianza del Proceso de Dirichlet, y, por lo tanto, una mayor concentraci\'on respecto a la distribuci\'on media $H$. 

Siguiendo la secuencia l\'ogica, si $\alpha \rightarrow \infty$, entonces $G(A_i) \rightarrow H(A_i)$ para cualquier elemento $A_i$ de la partici\'on. Es decir, $G \rightarrow H$ en distribuci\'on. Sin embargo, cabe aclarar que esto no es lo mismo que $G \rightarrow H$. Por un lado, $H$ puede ser una distribuci\'on de probabilidad continua, mientras que, como se ver\'a m\'as adelante, $G$ puede arrojar dos muestras iguales con probabilidad mayor a 0, por lo que es una distribuci\'on discreta.

\subsection{Distribuci\'on posterior}

Sea $G \sim DP(\alpha,H)$. Dado que $G$ es (aunque aleatoria) una distribuci\'on, es posible obtener realizaciones de ella. Sean $\phi_1,..., \phi_n$ una secuencia de realizaciones independientes de $G$, que toman valores dentro de su soporte $\Theta$. Sea de nuevo $A_1,...,A_r$ una partici\'on finita cualquiera del conjunto $\Theta$, y sea $n_k = |\{i: \phi_i \in A_k\}|$ el n\'umero de valores observados dentro del conjunto $A_k$. Por la propiedad conjugada entre la distribuci\'on de Dirichlet y la distribuci\'on Multinomial, se obtiene que
\begin{equation*}
   (G(A_1),...,G(A_r))|\phi_1,...,\phi_n \sim Dir(\alpha H(A_1) + n_1,...,\alpha H(A_r) + n_r). 
\end{equation*}

Es momento de analizar el par\'ametro $\alpha H(A_k) + n_k$ de la distribuci\'on de Dirichlet, correspondiente a $G(A_k|\phi_1,...,\phi_n)$, donde $k \in \{1,...,r\}$. Pero antes, es importante observar que es posible reescribir $n_k = \sum_{i=1}^n \delta_i(A_k)$, donde $\delta_i(A_k) = 1$ si $\phi_i \in A_k$, y $0$ en cualquier otro caso. 
\begin{equation*}
\begin{aligned}
    \alpha H(A_k) + n_k 
    &= \alpha H(A_k) + \sum_{i=1}^n \delta_i(A_k) \\
    &= (\alpha + n)
    \left[
        \frac{\alpha \times H(A_k) + n \times \frac{\sum_{i=1}^n \delta_i(A_k)}{n}}{\alpha + n}
    \right] \\
    &= \bar{\alpha} \bar{H}(A_k),
\end{aligned}
\end{equation*}
donde
\begin{equation*}
\begin{aligned}
    \bar{\alpha} &= \alpha + n \\
    \bar{H}(A_k) &=  
        \left(\frac{\alpha}{\alpha + n}\right)H(A_k) + 
        \left(\frac{n}{\alpha + n}\right)\frac{\sum_{i=1}^n \delta_i(A_k)}{n}.
\end{aligned}
\end{equation*}

Por lo tanto, $G|\phi_1,...,\phi_n \sim DP(\bar{\alpha},\bar{H})$. Es decir, la probabilidad posterior de $G$ sigue distribuy\'endose mediante un Proceso de Dirichlet, con par\'ametros actualizados. Asimismo, se puede interpretar a la distribuci\'on media posterior $\bar{H}$ como una mezcla entre la distribuci\'on media inicial, con peso proporcional al par\'ametro de concentraci\'on inicial $\alpha$, y la distribuci\'on emp\'irica de los datos, con peso proporcional al n\'umero de observaciones $n$. Por otro lado, el par\'ametro de concentaci\'on posterior es estrictamente m\'as grande que el inicial, por lo que a medida que crecen las $n$ observaciones, la varianza del Proceso de Dirichlet se reduce y el Proceso se concentra alrededor de la distribuci\'on emp\'rica. 

\subsection{Distribuci\'on predictiva}

Continuando con la idea de la secci\'on anterior de que ya se conoce el valor de $\phi_i,...,\phi_n$ realizaciones provenientes de la distribuci\'on aleatoria $G$, se desea hacer predicci\'on de la observaci\'on $\phi_{n+1}$, condicionada a los valores observados. As\'i,
\begin{equation*}
\begin{aligned}
   P(\phi_{n+1} \in A_k|\phi_1,...,\phi_n)
   &= \int P(\phi_{n+1} \in A_k|G) P(G|\phi_1,...,\phi_n) dG \\ 
   &= \int G(A_k) P(G|\phi_1,...,\phi_n) dG \\ 
   &= \mathbb{E}[G(A_k)|\phi_1,...,\phi_n] \\
   &= \bar{H}(A_k),
\end{aligned}    
\end{equation*}
es decir, 
\begin{equation*}
    \phi_{n+1}|\phi_1,...,\phi_n \sim 
    \left(\frac{\alpha}{\alpha + n}\right)H(\phi_{n+1}) + 
    \left(\frac{n}{\alpha + n}\right)\frac{\sum_{i=1}^n \delta_i(\phi_{n+1})}{n}.
\end{equation*}

Cabe resaltar que dicha distribuci\'on predictiva tiene puntos de masa localizados en $\phi_1,...,\phi_n$. Esto significa que la probabilidad de que $\phi_{n+1}$ tome un valor que ya ha sido observado es mayor a $0$, independientemente de la forma de $H$. Yendo a\'un m\'as all\'a, es posible darse cuenta que si se obtienen realizaciones infinitas de $G$, cualquier valor obtenido ser\'a repetido eventualmente, con probabilidad igual a $1$. Por lo tanto, $G$ es una distribuci\'on discreta.

\section{Problemas equivalentes y aplicaciones}

En esta secci\'on se revisar\'a la equivalencia entre los Procesos de Dirichlet y otros problemas famosos en la literatura, lo que facilitar\'a el entendimiento de la intiuici\'on que hay detr\'as, as\'i como la resoluci\'on y demostraci\'on de algunas propiedades pendientes.

\subsection{Esquema de urna de Blackwell-MacQueen}

Sea $\Theta$ un conjunto (finito o infinito) cuyos elementos son colores distintos al negro y al blanco, y donde cada color es distinto entre s\'i. Existe una m\'aquina llamada $H$ que cada que se le oprime \textit{Play} arroja de manera aleatoria una pelota con algún color perteneciente al conjunto $\Theta$, siguiendo una regla de probabilidad dada previamente. Se tienen 2 urnas: una llamada \textit{probabilidades}, que contiene $\alpha$ bolas negras. Otra llamada \textit{resultados}, que en un principio se encuentra vac\'ia. 

Se oprime \textit{Play} a la m\'aquina, y se obtiene una pelota, la cual se arroja a la urna \textit{resultados}. A $\phi_1$ se le aginar\'a el color de dicha pelota. Posteriormente se añade una pelota de color blanca a la urna \textit{probabilidades} y se pasa a la segunda ronda.  

Las siguientes rondas, por ejemplo la ronda $n+1$, comienza tomando al azar una pelota de la urna \textit{probabilidades}. Si el color de la pelota es negra (probabilidad proporcional a $\alpha$), se obtiene una nueva pelota de la m\'aquina $H$ y se repite lo sucedido en la primera ronda, incluyendo al asignar el color de la pelota a $\phi_{n+1}$. Si es blanca, se toma al azar una pelota de la urna \textit{resultados}, se asigna el color de esa pelota a $\phi_{n+1}$ y se regresa a la urna de \textit{resultados} esa misma pelota, as\'i como una nueva pintada del mismo color. En ambos casos, después de hacer lo antes mencionado, se introduce una nueva pelota blanca a la urna \textit{probabilidades} y se pasa a la siguiente ronda. 

Así, despu\'es de $n$ rondas, se obtiene la secuencia $\phi_1,...,\phi_n$. Es importante notar que cada $\phi_{k+1}$ es una variable aleatoria que depende de las $k$ anteriores, y cuya distribuci\'on es
\begin{equation*}
    \phi_{k+1}|\phi_1,...,\phi_k \sim 
    \left(\frac{\alpha}{\alpha + k}\right)H(\phi_{k+1}) + 
    \left(\frac{k}{\alpha + k}\right)\frac{\sum_{i=1}^k \delta_i(\phi_{k+1})}{k}.
\end{equation*}

La distribuci\'on conjunta de $\phi_1,...,\phi_n$ se puede obtener como
\begin{equation*}
    P(\phi_1,...,\phi_n) = 
    P(\phi_1)
    \prod_{i=2}^n
    P(\phi_i|\phi_1,...,\phi_{i-1})
\end{equation*}

Antes de continuar, es importante repasar una definici\'on. 

\begin{defin}
    Sea $\phi_1,...,\phi_n$, una secuencia de $n$ variables aleatorias, cuya distribuci\'on de probabilidad conjunta est\'a dada por $P(\phi_1,...,\phi_n)$. Sea $\psi$ una funci\'on biyectiva, que va de $\{1,...,n\} \rightarrow \{1,...,n\}$, es decir, una funci\'on que crea una permutaci\'on del conjunto $\{1,...,n\}$.  
    Entonces, se dice que $\phi_1,...,\phi_n$ es una \textbf{secuencia aleatoria infinitamente intercambiable} si se cumple que 
    \begin{equation*}
        P(\phi_1,...,\phi_n) = P(\phi_{\psi(1)},...,\phi_{\psi(n)}),
    \end{equation*}
    para cualquier permutaci\'on $\psi$.
\end{defin}

Regresando al juego de urnas, es importante observar que si bien $\phi_{k+1}$ es dependiente de las $k$ observaciones anteriores, esta dependencia s\'olo se da en t\'erminos de los valores observados previamente y la frecuencia de dichas observaciones, pero el orden en que hayan sido obtenidos no es relevante. Por lo tanto, es posible afirmar que $\phi_1,...,\phi_n$ es una secuencia aleatoria infinitamente intercambiable. Dicho esto, es conveniente recordar el \textbf{\textit{Teorema de represesentaci\'on general de de Finetti}}.\footnote{Una demostraci\'on de este teorema puede ser encontrada en \cite{Schervish_TheoryStats}.}

\newtheorem{theorem}{Teorema}

\begin{theorem}
    Sea $\phi_1, ...,\phi_n$ una secuencia aleatoria infinitamente intercambiable de valores reales. Entonces existe una distribuci\'on de probabilidad $G$ sobre $\mathcal{F}$, el espacio de todas las distribuciones, de forma que la probabilidad conjunta de $\phi_1, ...,\phi_n$ se puede expresar como
    \begin{equation*}
        P(\phi_1, ...,\phi_n) =
        \int_{\mathcal{F}}\left[\prod_{k=1}^n G(\phi_k)\right]dP(G),
    \end{equation*}
    con
    \begin{equation*}
        P(G) = \lim_{n \to \infty} P(G_n),
    \end{equation*}
    donde $P(G_n)$ es una funci\'on de distribuci\'on evaluada en la funci\'on de distribuci\'on emp\'irica definida por
    \begin{equation*}
        G_n = \frac{1}{n} \sum_{i=1}^n I(y_i \leq y).
    \end{equation*}
    En otras palabras, el Teorema de de Finetti dice que existe una distribuci\'on $G$ tal que $\phi_1, ...,\phi_n$ son condicionalmente independientes, dada dicha $G$. A su vez dicha $G$ es aleatoria y sigue una distribuci\'on $P(G)$.
\end{theorem}

Una vez dicho esto, y sean $\phi_1,...,\phi_n$ una secuencia de colores obtenida con la rutina de esta secci\'on, es posible darse cuenta que cada $\phi_k \sim G$. Adem\'as $P(G) = DP(\alpha,H)$, seg\'un lo visto en la secci\'on anterior. Con esto, queda demostrada la existencia de los Procesos de Dirichlet.

\subsection{Proceso estoc\'astico del restaurante chino}

Sean $\phi_1,...,\phi_n$ una secuencia de realizaciones de $G$, con $G \sim DP(\alpha,H)$. Recordando lo mencionado anteriormente, cada valor obtenido tiene una probabilidad mayor a $0$ de ser repetido en una nueva observaci\'on. 

Sean $\phi_1^*,...,\phi_m^*$ los $m$ valores \'unicos observados, y sea $n_k^*$ sea el n\'umero de veces que se repite cada valor $\phi_k^*$. Entonces, la distribuci\'on predictiva se puede reescribir como
\begin{equation*}
    \phi_{n+1}|\phi_1,...,\phi_n \sim 
    \left(\frac{\alpha}{\alpha + n}\right)H(\phi_{n+1}) + 
    \left(\frac{n}{\alpha + n}\right)\frac{\sum_{k=1}^m n_k^*\delta_{\phi_k^*}(\phi_{n+1})}{n}.
\end{equation*}

A partir de este momento se definir\'a como el \textit{cluster} $\Phi_k^*$ al conjunto cuyos elementos son todos los $\phi_i 's$ id\'enticos que tomen el valor $\phi_k^*$. Es inmediato observar que la probabilidad de que una nueva observaci\'on $\phi_{n+1}$ se ubique dentro del \textit{cluster} $\Phi_k^*$ es proporcional a $n_k^*$. Es decir, se da el fen\'omeno de \textit{los ricos se vuelven m\'as ricos}, ya que entre mayor sea el n\'umero de elementos de un \textit{cluster}, mayor ser\'a la probabilidad de que una nueva observaci\'on sea parte de \'el.

As\'i, queda inducida una partici\'on sobre el conjunto $N = \{1,...,n\}$, debido a que cada uno de dichos n\'umeros naturales pertenece a un, y s\'olo un, $\Phi_k^*$. Adem\'as, el \textit{cluster} al que pertenecer\'a cada uno es aleatorio, por lo que la partici\'on inducida tambi\'en es aleatoria, y encapsula todas las propiedades de los Procesos de Dirichlet.

Para ver c\'omo sucede esto, \'unicamente hay invertir el proceso generador. En este caso, primero se obtiene de manera aleatoria una partici\'on del conjunto $N$ en \textit{clusters} $\Phi_1^*,...,\Phi_m^*$. Despu\'es, para cada $\Phi_k^*$ se encuentra su respectivo valor mediante una realizaci\'on de $\phi_k^* \sim H$, y se asigna $\phi_i = \phi_k^*$, para toda $i \in \Phi_k^*$.

La distribuci\'on sobre las particiones ha sido nombrada el \textit{Proceso estoc\'astico del restaurante chino}, y es un problema que ha sido estudiado de manera independiente a los Procesos de Dirichlet, siendo descubierta posteriormente su equivalencia. Su nombre lo toma de la siguiente met\'afora.

Se supone un restaurante con infinito n\'umero de mesas e infinitas sillas en cada una de ellas. El primero consumidor entra y se sienta en la mesa $\Phi_1^*$. El segundo entra y con probabilidad $\frac{1}{\alpha + 1}$ se sienta en la misma mesa $\Phi_1^*$ del consumidor anterior, mientras que con probabilidad $\frac{\alpha}{\alpha+1}$ se sienta en una nueva mesa $\Phi_2^*$. 

En general, despu\'es de que han entrado $n$ personas, han sido ocupadas $m$ mesas. Sea $n_k^*$ el n\'umero de personas que est\'an sentadas en la mesa $\Phi_k^*$, una nueva persona $n+1$ se sienta en una mesa con las siguientes probabilidades:
\begin{equation*}
\begin{aligned}
   P(n+1 \in \Phi_{m+1}^*) &= \frac{\alpha}{\alpha + n},\\
    P(n+1 \in \Phi_{k}^*) &= \frac{n_k^*}{\alpha + n},
\end{aligned}
\end{equation*}
siendo $\Phi_{m+1}^*$ una mesa que a\'un no ha sido ocupada.

Para conectar esta met\'afora con los Procesos de Dirichlet, se podr\'ia pensar que todos los integrantes de cada mesa comer\'an el mismo platillo, mismo que ser\'ia elegido aleatoriamente mediante la distribuci\'on H, entre un infinito n\'umero de platillos.

\subsection{Proceso estoc\'astico de rompimiento de un palo}
Es importante recordar que una realizaci\'on $G$ de un Proceso de Dirichlet es una distribuci\'on discreta con probabilidad $1$, debido a que toda muestra tiene probabilidad mayor a $0$ de ser repetida. Por lo tanto, se puede expresar a $G$ como una suma de centros de masa, de la siguiente manera:
\begin{equation*}
\begin{aligned}
   \phi_k^* &\sim H, \\
   G(\phi) &= \sum_{k=1}^\infty \pi_k \delta_{\phi_k^*}(\phi),
\end{aligned}
\end{equation*}
siendo $\pi_k$ la probabilidad de ocurrencia de $\phi_k$.

Dicha probabilidad de ocurrencia ser\'a generada con la siguiente met\'afora. Se piensa un palo de longitud 1. Se genera una n\'umero aleatorio $\beta_1 \sim Beta(1,\alpha)$, mismo que estar\'a en el intervalo $(0,1)$. Esa ser\'a la magnitud del pedazo que ser\'a separado del palo de longitud 1, y le ser\'a asignado a $\pi_1 = \beta_1$. As\'i, quedar\'a un palo de magnitud $(1-\beta_1)$ a repartir. Posteriormente se vuelve a generar un n\'umero aleatorio $\beta_2 \sim Beta(1,\alpha)$, que representar\'a la proporci\'on del palo restante que le ser\'a asignada a $\pi_2$. Es decir, $\pi_2 = \beta_2(1-\beta_1)$. En general, para $k \geq 2$,
\begin{equation*}
\begin{aligned}
   \beta_k &\sim Beta(1,\alpha),\\
   \pi_k &= \beta_k \prod_{i=1}^{k-1}(1 - \beta_i).
\end{aligned}
\end{equation*}
Dada su construcci\'on, es inmediato darse cuenta que $\sum_{k=1}^\infty \pi_k = 1$. Algunas ocasiones se nombra a esta distribuci\'on $\pi \sim GEM(\alpha)$, en honor a Griffiths, Engen y McCloskey.

\subsection{Modelo de mezclas infinitas de Dirichlet}

Sean $\{y_1,...,y_n\}$ un conjunto de observaciones con distribuci\'on $F$, condicionalmente independientes, y que se suponen vienen del \textit{Modelo de mezclas de Dirichlet}:
\begin{equation*}
\begin{aligned}
   y_i | \phi_i &\sim F(y_i | \phi_i), \\
   \phi_i | G &\sim G(\phi_i), \\
   G | \alpha, H &\sim DP(\alpha,H).
\end{aligned}
\end{equation*}
Se dice que este es un \textit{modelo de mezclas} debido a que existen $y_i's$ que comparten un mismo valor para $\phi_i$ (por la propiedad discreta de $G$), y entonces estas $y_i's$ pueden ser consideradas pertenecientes a una misma subpoblaci\'on.

Es posible reescribir este modelo usando la equivalencia entre los Procesos de Dirichlet y el Proceso estoc\'astico de rompimiento de un palo, visto anteriormente. Sea $z_i$ el \textit{cluster} al que pertenece $y_i$ entre los $\Phi_1^*,\Phi_2^*,...$ posibles, se tiene entonces que $P(z_i = \Phi_k^*) = \pi_k$. Y si $\phi_k^*$ es el valor que comparten los miembros de $\Phi_k^*$, se usar\'a la notaci\'on $\phi_{z_i} = \phi_k^*$, cuando $z_i = \Phi_k^*$. Por lo tanto, el modelo se puede ahora escribir como
\begin{equation*}
\begin{aligned}
   y_i | z_i, \phi_k^* &\sim F(y_i | \phi_{z_i}), \\
   z_i | \pi &\sim Mult(\pi), \\
   \pi | \alpha &\sim GEM(\alpha), \\
   \phi_k^* | H &\sim H.
\end{aligned}
\end{equation*}

De esta manera, el Modelo de mezclas de Dirichlet es un modelo de mezclas infinitas, debido a que tiene un n\'umero infinito numerable de posibles \textit{clusters}, pero donde intuitivamente la importancia realmente recae s\'olo en aquellos que tienen un peso $\pi$ posterior mayor a cierto umbral, pero que son detectados hasta despu\'es de observar los datos; a diferencia de los modelos de mezclas finitas, que ya tienen un n\'umero de \textit{clusters} definidos previamente.

\newpage

\chapter{Regresi\'on sobre cuantiles}

\section{Motivaci\'on}

Cuando surgi\'o entre la comunidad estad\'istica el problema de \textit{regresi\'on sobre cuantiles}, inicialmente fue modelado bajo un enfoque no bayesiano, como se describe en \cite{Yu_BayQuantReg}. 

Sea $q_p$ el cuantil \textit{p}-\'esimo de $X$, es decir, $P(X \leq q_p) = p$. Entonces el cuantil \textit{p}-\'esimo condicional de $y$, dado $x$, se supondr\'a es posible escribirlo como
\begin{equation*}
    q_p(y|x) = x^T\beta(p), 
\end{equation*}
donde $\beta(p)$ es el vector de coeficientes, dependientes de $p$.

Se define una funci\'on de p\'erdida como 
\begin{equation*}
    \rho_p(u) = u \times (p - I_{(u<0)}),
\end{equation*}
misma que se puede reescribir como 
\begin{equation*}
    \rho_p(u) = u \times [pI_{(u>0)} - (1-p) I_{(u<0)})],
\end{equation*}
o
\begin{equation*}
    \rho_p(u) = 
    \frac{|u| + (2p-1)u}{2}.
\end{equation*}

Siguiendo este orden de ideas, se puede demostrar que para el problema de minimizaci\'on
\begin{equation*}
    \begin{aligned}
    \underset{q_p}{\text{min}} \text{ }
    \mathbb{E} [\rho_p(y_i - q_p)],
    \end{aligned}
\end{equation*}
la soluci\'on $q_p^*$ cumple que $P(X \leq q_p^*) = p$.

As\'i, la primera idea para resolver el problema de \textit{regresi\'on sobre cuantiles} fue resolver el problema de minimizaci\'on
\begin{equation*}
\begin{aligned}
\underset{\beta(p)}{\text{min}}
\sum_i \rho_p(y_i - x_i^T \beta(p)).
\end{aligned}
\end{equation*}

Posteriormente, Koenker \& Bassett (1978) retomaron esta idea, aplic\'andola en el paradigma bayesiano. 

\begin{defin}
    Se dice que una variable aleatoria U sigue una distribuci\'on asim\'etrica de Laplace si su funci\'on de densidad se escribe como
    \begin{equation*}
        f_p(u|\mu,\sigma) = 
        \frac{p(1-p)}{\sigma}
        exp\left[
        -\rho_p
        \left(
        \frac{u-\mu}{\sigma}
        \right)
        \right],
    \end{equation*}
con $\mu$ par\'ametro de localizaci\'on y $\sigma$ par\'ametro de escala.
\end{defin}

Es f\'acilmente demostrable que el anterior problema de minimizaci\'on de los errores es equivalente a maximizar la funci\'on de verosimilitud formada como producto de densidades independientes asim\'etricas de Laplace.

Si bien esto represent\'o un gran avance, a\'un queda la posibilidad de retomar estas ideas y crear modelos m\'as precisos. La intenci\'on de esta tesis es encontrar un modelo para la \textit{regresi\'on sobre cuantiles} que sea completamente bayesiano y no param\'etrico, con la intenci\'on de poder representar distribuciones m\'as complejas.

\section[Modelos]{
    Modelos
    \footnote{Los primeros dos son retomados de \cite{Kottas_SemiparamQuantReg}, y el tercero de \cite{Kottas_NotParamQuantReg}.}
}

Esta secci\'on buscar\'a desarrollar modelos que tomen en cuenta los aprendizajes previos, para realizar an\'alisis de \textit{regresi\'on sobre cuantiles}. A lo largo de ella se supondr\'a lo que se anuncia a continuaci\'on.

\begin{defin}
    El \textbf{modelo de regresi\'on del cuantil \textit{p}-\'esimo}, para una variable de respuesta $y \in \mathbb{R}$, y un conjunto de covariables $x \in \mathbb{R}^n$, ser\'a aquel que se pueda escribir como
    \begin{equation*}
        y = f(x) + \varepsilon,
    \end{equation*}
    donde $h: \mathbb{R}^{n} \rightarrow \mathbb{R}$, y $\varepsilon$ es el error aleatorio, independiente de $x$, y cuyo cuantil \textit{p}-\'esimo es igual a $0$. Es decir, si $h_p$ es la funci\'on de densidad de $\varepsilon$, se tiene que
    \begin{equation*}
        \int_{-\infty}^0 h_p(\varepsilon) d\epsilon = p.
    \end{equation*}
\end{defin}

De la definici\'on anterior es posible darse cuenta que el problema de regresi\'on por cuantiles se puede reinterpretar como el problema de encontrar la forma de $f(x)$ y la de $h_p(\varepsilon)$.

Es inmediato darse cuenta que $h_p(\varepsilon)$ tendr\'a una forma sim\'etrica si, y s\'olo si, $p = 0.5$. Es decir, sera sim\'etrica \'unicamente para el modelo de \textit{regresi\'on sobre la mediana}, y asim\'etrica en cualquier otro caso. Intuitivamente, esto provocar\'a que el modelo espere una proporci\'on de errores negativos similar a $p$, y de errores positivos similar a $1-p$, que coincide con lo buscado al realizar un modelo de este tipo.

A continuaci\'on se plantean diversos modelos que realizan supuestos adicionales acerca de la forma de $f(x)$ y de $h_p(\varepsilon)$, para resolver el problema.

\subsection{Mezcla asim\'etrica de densidades de Laplace}

En estos primeros modelos se supondr\'a una forma lineal para $f$, es decir, $f$ se podr\'a escribir como $f(x) = x^T \beta$. Por otro lado, se supondr\'a como forma param\'etrica de $h_p$ a la familia de distribuciones asim\'etricas de Laplace, sin par\'ametro de localizaci\'on, y cuya densidad se escribe como
\begin{equation*}
    w_p^{AL}(\varepsilon,\sigma) = 
    \frac{p(1-p)}{\sigma}
    exp \left(
    - \frac{|\varepsilon| + (2p - 1) \varepsilon}{2 \sigma}
    \right),
\end{equation*}
con par\'ametro de escala $\sigma \in \mathbb{R}^+$ y $0 < p <1$. Dado que es poco com\'un que se conozca el valor de $\sigma$, se recurrir\'a a un modelo de mezclas infinitas de Dirichlet (visto en el cap\'itulo anterior), con par\'ametro de concentraci\'on $\alpha$, y distribuci\'on media $H$, con soporte en $\mathbb{R}^+$. Se define entonces la funci\'on de densidad del error aleatorio como
\begin{equation*}
\begin{aligned}
    h_p^{AL}(\varepsilon|G) &= \int w_p^{AL}(\varepsilon|\sigma)dG(\sigma), \\
    G &\sim DP(\alpha,H).
\end{aligned}
\end{equation*}

Cabe resaltar que a pesar de la mezcla, se sigue cumpliendo la condici\'on de que $\int_{-\infty}^0 h_p^{AL}(\varepsilon|G) d\varepsilon = p$, para cualquier $G$. En cuanto a los par\'ametros del Proceso de Dirichlet, se tomar\'a una $\alpha > 0$, y $H$ ser\'a una Gamma-Inversa, con par\'ametro de forma $c > 0$, y de raz\'on $d > 0$. Siguiendo este orden de ideas, se puede reescribir el modelo como
\begin{equation*}
\begin{aligned}
    y-x^T \beta | \beta, \sigma &\sim w_p^{AL}(\varepsilon | \sigma), \\
    \beta &\sim \mathcal{N}(\beta | m,v), \\
    \sigma | G &\sim G(\sigma), \\
    G | \alpha, c, d &\sim DP(\alpha, H(c,d))
\end{aligned}
\end{equation*}

\subsection{Mezcla de densidades uniformes, con $f(x)$ lineal}

Como se puede corroborar en \cite{Pavlides_NonparamMixUnifDens}, para cualquier funci\'on de densidad $h(\cdot)$ no creciente y con soporte en $R^{+}$, existe una distribuci\'on $G$, tal que
\begin{equation*}
    h(x|G) = \int \theta^{-1} I_{[0,\theta)} (x) dG(\theta).
\end{equation*}

En otras palabras, $h(\cdot)$ puede ser escrita como una mezcla de densidades uniformes.

Siguiendo este orden de ideas, y suponiendo que $G \sim (\alpha, H)$, dicho resultado puede ser usado para definir la parte positiva de la funci\'on de densidad $h_p(\varepsilon)$. Por otro lado, la parte negativa se puede definir de la misma manera usando un signo negativo. Entonces, se tiene que
\begin{equation*}
    h_p(\varepsilon) = \int \int w_p^{DU}(\varepsilon|\sigma_1,\sigma_2)dG_1(\sigma_1)dG_2(\sigma_2),
\end{equation*}
donde $G_1$ y $G_2$ representan a las partes negativas y positivas, respectivamente.

Adem\'as, vale la pena recordar que el cuantil \textit{p}-\'esimo tiene que ser igual a $0$, por lo que el peso de cada uno de los dos lados estar\'a dado por
\begin{equation*}
    w_p^{DU}(\varepsilon|\sigma_1,\sigma_2) =
    \frac{p}{\sigma_1} I_{(-\sigma_1,0)}(\varepsilon) + 
    \frac{(1-p)}{\sigma_2} I_{(0,\sigma_2)}(\varepsilon).
\end{equation*}

De esta manera se tiene un modelo m\'as general que el de Mezcla asim\'etrica de densidades de Laplace, ya que el \'unico supuesto es que la moda est\'a en 0 y la densidad es no creciente hacia las colas. En resumen, el modelo obtenido es
\begin{equation*}
\begin{aligned}
    y-x^T \beta | \beta, \sigma &\sim w_p^{DU}(\varepsilon | \sigma_1, \sigma_2), \\
    \beta &\sim \mathcal{N}(\beta | m,v), \\
    \sigma_r | G_r &\sim G_r(\sigma_r),\\
    G_r | \alpha_r, c_r, d_r &\sim DP(\alpha_r, H(c_r,d_r)), \\
    r &\in \{1,2\}.
\end{aligned}
\end{equation*}

\subsection{Mezcla de densidades uniformes, con Procesos Gaussianos}

Hasta ahora, se ha supuesto que $f(x)$ es lineal. Para darle mayor flexibilidad al modelo, ahora se supondr\'a que $f(x) \sim \mathcal{GP}(m(x),k(x,x))$, con $m$ funci\'on de medias dada por el modelador, y $k$ funci\'on de covarianzas \textit{exponencial cuadrada}, es decir,
\begin{equation*}
    k(x,x'|\lambda^2) = 
    exp\left(-\frac{1}{2}
    \frac{\norm{x-x'}_2^2}{\lambda^2}
    \right),
\end{equation*}
donde, a su vez, $\lambda \sim GI(\lambda_\alpha,\lambda_\beta)$.

Añadiendo esta idea al modelo anterior, se obtiene
\begin{equation*}
\begin{aligned}
    y-f(x) | f(x), \sigma &\sim w_p^{DU}(\varepsilon | \sigma_1, \sigma_2), \\
    f(x) &\sim \mathcal{GP}(m(x),k(x,x|\lambda)), \\
    \lambda &\sim GI(\lambda_\alpha,\lambda_\beta), \\
    \sigma_r | G_r &\sim G_r(\sigma_r),\\
    G_r | \alpha_r, c_r, d_r &\sim DP(\alpha_r, H(c_r,d_r)), \\
    r &\in \{1,2\}.
\end{aligned}
\end{equation*}

\newpage

%%%%%%%%%%%%%%%%%%%%%%%%%%%%%%%%%%%%%%%%%%%%%%%%%%%%%%%%%%%%%
%%    Bibliografia
%%%%%%%%%%%%%%%%%%%%%%%%%%%%%%%%%%%%%%%%%%%%%%%%%%%%%%%%%%%%%
\nocite{*} %Even non-cited BibTeX-Entries will be shown.
\bibliographystyle{authordate1} %Style of Bibliography: plain / apalike / amsalpha / ...
\bibliography{Bibliography} %You need a file 'literature.bib' for this.

%%%%%%%%%%%%%%%%%%%%%%%%%%%%%%%%%%%%%%%%%%%%%%%%%%%%%%%%%%%%%
%%  Apendices
%%%%%%%%%%%%%%%%%%%%%%%%%%%%%%%%%%%%%%%%%%%%%%%%%%%%%%%%%%%%%
%\appendix

%%%%%%%%%%%%%%%%%%%%%%%%%%%%%%%%%%%%%%%%%%%%%%%%%%%%%%%%%%%%%%
%% ==> Some hints are following:

\chapter{Some small hints}\label{hints}

\section{German Umlauts and other Language Specific Characters}\label{umlauts}
You can type german umlauts like '�', '�', or '�' directly in this file.
This is also true for other language specific characters like '�', '�' etc.

There are problems with automatic hyphenation when using language
specific characters and OT1-encoded fonts. In this case, use a
T1-encoded Type1-font like the Latin Modern font family (\verb#\usepackage{lmodern}#).


\section{References}\label{references}
Using the commands \verb#\label{name}# and \verb#\ref{name}# you are able
to use references in your document. Advantage: You do not need to think
about numerations, because \LaTeX\ is doing that for you.

For example, in section \ref{dividing} on page \pageref{dividing} hints for
dividing large documents are given.

Certainly, references do also work for tables, figures, formulas\ldots

Please notice, that \LaTeX\ usually needs more than one run (mostly 2) to
resolve those references correctly.


\section{Dividing Large Documents}\label{dividing}
You can divide your \LaTeX-Document into an arbitrary number of \TeX-Files
to avoid too big and therefore unhandy files (e.g. one file for every chapter).

For this, you insert in your main file (this one) for every subfile
the command '\verb#\input{subfile}#'. This leads to the same behavior
as if the content of the subfile would be at the place of the \verb#\input#-Command.

%% <== End of hints
%%%%%%%%%%%%%%%%%%%%%%%%%%%%%%%%%%%%%%%%%%%%%%%%%%%%%%%%%%%%%


\end{document}

