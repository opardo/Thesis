\chapter[Conclusiones y trabajo futuro]{Conclusiones y trabajo futuro}

Si bien los modelos de regresi\'on a la media han sido de mucha utilidad en las \'ultimas d\'ecadas, principalmente cuando el poder computacional era menor, es importante recalcar que actualmente existen contextos en los que resultan insuficientes, tanto porque se quiere estudiar qu\'e tan factible es un valor at\'ipico, o porque se necesita modelar alg\'un fen\'omeno asim\'etrico, por mencionar algunos ejemplos.

De manera similar, la aproximaci\'on lineal y la distribuci\'on Normal del error han sido fundamentales para que los modelos de regresi\'on hayan proliferado en una gran cantidad de industrias, tanto por su interpretabilidad, como por su bajo costo. Pero es imposible ignorar que \'unicamente representan un subconjunto del universo infinito de funciones y dispersiones posibles. Crear modelos que permitan una mayor flexibilidad, como los surgidos de los m\'etodos no param\'etricos, acercar\'an m\'as a la Estad\'istica a una representaci\'on certera de la realidad.

Asimismo, utilizar el paradigma Bayesiano para realizar este tipo de modelado tiene la ventaja de poder introducir informaci\'on de las y los expertos en el fen\'omeno a estudiar, as\'i como ponderar cu\'ando fiarse m\'as de los datos y cu\'ando de dicho conocimiento. Adem\'as, de fondo tiene una construcci\'on axiom\'atica, que todo aquel que disfrute de la formalidad en las Matem\'aticas, valorar\'a.

Si bien estos avances son significativos, a\'un existe mucho que explorar respecto a lo expuesto en este trabajo. Por ejemplo, se podr\'ia plantear una descomposici\'on de la funci\'on aproximada del cuantil en muchos procesos Gaussianos, uno por covariable, situaci\'on que podr\'a brindar un mayor peso a aquellas covariables que en efecto sean m\'as significativas para explicar el fen\'omeno en cuesti\'on. 

Por otro lado, la inclusi\'on de un parametro de rango que regule din\'amicamente la relaci\'on entre la distancia y la covarianza entre observaciones, de acuerdo al fen\'omeno y a las covariables usadas, brindar\'a mayor flexibilidad, y por ende, un mejor ajuste al modelo.

\newpage 