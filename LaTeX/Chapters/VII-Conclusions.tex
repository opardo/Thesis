\chapter[Conclusiones y trabajo futuro]{Conclusiones y trabajo futuro}

Si bien los modelos de regresi\'on a la media han sido de mucha utilidad en las \'ultimas d\'ecadas, principalmente cuando el poder computacional era menor, es importante recalcar que actualmente existen contextos en los que resultan insuficientes, tanto porque se quiere estudiar qu\'e tan factible es un valor at\'ipico, o porque se necesita modelar alg\'un fen\'omeno asim\'etrico, por mencionar algunos ejemplos.

De manera similar, la aproximaci\'on lineal y la distribuci\'on Normal del error han sido fundamentales para que los modelos de regresi\'on hayan proliferado en una gran cantidad de industrias, tanto por su interpretabilidad, como por su bajo costo. Pero es imposible ignorar que \'unicamente representan un subconjunto del universo infinito de funciones y dispersiones posibles. Crear modelos que permitan una mayor flexibilidad, como los surgidos de los m\'etodos no param\'etricos, acercar\'an m\'as a la Estad\'istica a una representaci\'on certera de la realidad.

Asimismo, utilizar el paradigma Bayesiano para realizar este tipo de modelado tiene la ventaja de poder introducir informaci\'on de las y los expertos en el fen\'omeno a estudiar, as\'i como ponderar cu\'ando fiarse m\'as de los datos y cu\'ando de dicho conocimiento. Adem\'as, de fondo tiene una construcci\'on axiom\'atica, que todo aquel que disfrute de la formalidad en las Matem\'aticas, valorar\'a.

Un reto importante que present\'o este trabajo fue la realizaci\'on del paquete para implementarlo en R. Esto debido a que se tuvieron que realizar funciones lo suficientemente generales para funcionar con los insumos b\'asicos definidos. Si bien el tiempo que tarda en correr es elevado, es entendible porque la simulaci\'on de las cadenas de Markov no se puede hacer en paralelo, y para lograr una mejor estimaci\'on, requiere incrementar el n\'umero de iteraciones. Por ello el paquete deja ese par\'ametro a decisi\'on del modelador, de tal manera que \'el o ella eval\'ue si prefiere precisi\'on o velocidad.

Si bien estos avances son significativos, a\'un existe mucho que explorar respecto a lo expuesto en este trabajo. Por ejemplo, el modelo planteado en este trabajo no es capaz de darle un peso distinto a cada variable de entrada, sino que las toma por igual al momento de calcular la distancia entre observaciones. Para mejorar esta situaci\'on se podr\'ia plantear una descomposici\'on de la funci\'on estimada del cuantil en muchos procesos Gaussianos, uno por covariable, lo que brindar\'ia un mayor peso a aquellas covariables que en efecto sean m\'as significativas para explicar el fen\'omeno en cuesti\'on. 

Adem\'as, ser\'ia conveniente la inclusi\'on de un par\'ametro de rango que regule din\'amicamente la relaci\'on entre la distancia y la covarianza entre observaciones. Por ejemplo, a\'un cuando est\'en estandarizados los datos, una distancia de 1 podr\'ia significar una covarianza grande entre observaciones para alguna covariable o fen\'omeno, pero covarianza casi nula para otro. Lograr implementar este par\'ametro din\'amico brindar\'a mayor flexibilidad, y por ende, un mejor ajuste al modelo.

\newpage 