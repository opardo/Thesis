\chapter{Introducci\'on}

Detr\'as de cualquier modelo de regresi\'on la intenci\'on es entender alguna caracter\'istica asociada con una variable aleatoria, en funci\'on de un conjunto de variables potencialmente explicativas o predictivas para tal caracter\'istica. Ha sido com\'un resumir esta dependencia mediante alguna medida de tendencia central, condicionada a los valores de las covariables.

La medida de tendencia central tradicionalmente usada ha sido la media, dando lugar a los modelos de \textit{regresi\'on a la media}, con sus variantes lineal y no lineal, simple y m\'ultiple, con error normal y no normal. Este tipo de modelos tiene un buen n\'umero de ventajas, entre las que destacan el bajo costo de estimaci\'on y la facilidad de interpretaci\'on de sus par\'ametros (cuando el modelo es param\'etrico). Sin embargo, como mencionan \cite{Hao_FrecQuantReg}, tienen tres grandes limitaciones, que a continuaci\'on se mencionan.

La primera es que, generalmente, la estimaci\'on del modelo busca minimizar la diferencia entre el valor esperado te\'orico y la media observada. Por lo tanto, la inferencia de los valores lejanos a la media, que suelen ser de inter\'es en ciertos contextos, como los seguros o las finanzas, puede ser inexacta.

La segunda es que algunos fen\'omenos de estudio tienen distribuciones de colas pesadas, principalmente en las ciencias sociales. Esto da lugar a valores at\'ipicos, mismos que pueden sesgar estimaciones de la media, mientras prácticamente no afectan a otros estadísticos, como la mediana.

La tercera es que al focalizar la intervenci\'on de las variables explicativas \'unicamente en la media, y darle una distribución predefinida al error, se suelen dejar de lado caracter\'isticas del fen\'omeno que se est\'a estudiando. Un ejemplo es cuando se supone simetr\'ia o asimetr\'ia para todos los errores, pero en los datos se dan ambas condiciones, dependiendo del valor de las variables explicativas. 

Debido a esto, desde mitades del siglo XVIII han surgido alternativas a este tipo de modelos. Seg\'un \cite{Hao_FrecQuantReg}, la primera conocida data de 1760, cuando el jesuita croata Rudjer Josip Boscovich visit\'o Londres en b\'usqueda de consejo computacional para su novedoso modelo de \textit{regresi\’on a la mediana}. De nueva cuenta se busc\'o una medida de tendencia central, pero con otras bondades. Por ejemplo, ser una mejor medida informativa para distribuciones asim\'etricas y menos susceptible a valores at\'ipicos. 

As\'i como los modelos de regresi\'on a la media son com\'unmente relacionados con la minimizaci\'on de los errores cuadr\'aticos, los modelos de regresi\'on a la mediana lo son con la minimizaci\'on de los errores absolutos. Debido a la no diferenciabilidad, tuvieron que pasar muchos años para que lograran ser viables, hasta que el poder computacional y los algoritmos de programaci\'on lineal lo permitieron.

Cabe recordar que, a grandes rasgos y sin dar a\'un una definici\'on formal, el cuantil $p$-\'esimo es aquel valor tal que el $p \times 100\%$ de los valores est\'an por debajo de \'el, y el $(1-p)\times 100\%$, por encima. As\'i, la mediana es un caso particular de un cuantil, espec\'ificamente el $0.5$-\'esimo. Esto abre la idea de que otros cuantiles tambi\'en podr\'ian ser modelados en funci\'on de las covariables y no necesariamente tienen que ser una medida de tendencia central. 

Los \textit{modelos de regresi\'on sobre cuantiles} fueron introducidos por \cite{Koenker_QuantReg}, y han permitido concentrarse en valores de inter\'es para los modeladores, sin importar que est\'en alejados de la media. Adem\'as, el c\'alculo de diversos cuantiles para un mismo fen\'omeno ha permitido entender mejor la forma y propiedades de las distribuciones condicionales de la variable de respuesta.

En el paradigma Bayesiano el desarrollo de este tipo de modelos ha sido posterior al de otros paradigmas, y a\'un hay mucho por explorar. \cite{Walker_BayesAccFail}, \cite{Kottas_BaySemiparamMed} y \cite{Hanson_PolyaTrees} desarrollaron modelos para la mediana, suponiendo una distribuci\'on no param\'etrica del error. \cite{Yu_BayQuantReg} y \cite{Tsionas_BayQuantInf} desarrollaron inferencia param\'etrica, basados en la distribuci\'on asim\'etrica de Laplace para los errores. Por otro lado, \cite{Lavine_LikeQuant} y \cite{Dunson_ApproxBayes} usaron una perspectiva distinta y propusieron una aproximaci\'on de la verosimilitud para cuantiles.

Por la misma naturaleza compleja del nuevo desaf\'io, los modelos de regresi\'on sobre cuantiles han retomado un concepto que tambi\'en ha tomado auge en los de regresi\'on a la media: la distribuci\'on no param\'etrica de los errores, misma que generaliza la idea tradicional del error normal. 

Desafortunadamente, a diferencia de los modelos de regresi\'on a la media, que ya han propuesto alternativas para este tema, los modelos Bayesianos de regresi\'on sobre cuantiles han hecho muy poco por romper con la relaci\'on lineal en los par\'ametros entre la variable de respuesta y las covariables. Aquellos que lo han logrado, han tenido que recurrir a estimaciones no probabil\'isticas o no Bayesianas, para resolver alguna parte del problema.

Esta tesis tiene la finalidad de sustituir simult\'aneamente las ideas tradicionales de regresi\'on a la media, linearidad y distribuci\'on param\'etrica de los errores, con un enfoque m\'as flexible y totalmente probabil\'istico. Para ello, rescata las ideas de \cite{Kottas_NotParamQuantReg} y \cite{Kottas_SemiparamQuantReg}, proponiendo un modelo Bayesiano y no param\'etrico, \'util en el contexto de regresi\'on sobre cuantiles.

\newpage