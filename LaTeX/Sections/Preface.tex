\pagestyle{plain}
\chapter*{Prefacio}
\markboth{PREFACIO}{PREFACIO} % encabezado 

%%  Contenido
\addtocontents{toc}{\protect\vspace*{\baselineskip}}

El tema de esta tesis es describir un modelo de regresi\'on sobre cuantiles, debido a diversas bondades que presenta sobre el tradicional an\'alisis de regresi\'on a la media. Adem\'as, el modelo tiene una construcci\'on probabil\'istica realizada a través del paradigma Bayesiano y presenta una gran flexibilidad, al contar con componentes no param\'etricos. Asimismo, en este trabajo se abordan los modelos tradicionales de regresi\'on, para entender las desventajas contrarrestadas por el nuevo modelo.

El cap\'itulo 1 describe la importancia de las aproximaciones distintas al modelo tradicional de regresi\'on (a la media, lineal y con error normal), así como la evoluci\'on hist\'orica de estos modelos alternativos. El cap\'itulo 2 introduce al paradigma Bayesiano y sus fundamentos generales. El cap\'itulo 3 se centra en los modelos Bayesianos tradicionales de regresi\'on, tanto a la media, como sobre cuantiles. El cap\'itulo 4 plantea la especificaciones no param\'etricas para la tendencia y el error. El cap\'itulo 5 define el modelo propuesto por esta tesis y explica el algoritmo necesario para realizar inferencia y predicci\'on. El cap\'itulo 6 muestra simulaciones y aplicaciones del modelo. Finalmente, el cap\'itulo 7 hace referencia a las conclusiones de esta tesis, adem\'as de describir el trabajo futuro que se podr\'ia desarrollar.