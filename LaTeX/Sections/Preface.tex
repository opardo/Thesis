\pagestyle{plain}
\chapter*{Prefacio}
\markboth{PREFACIO}{PREFACIO} % encabezado 

%%  Contenido
\addtocontents{toc}{\protect\vspace*{\baselineskip}}

El tema de esta tesis es describir un modelo de regresi\'on sobre cuantiles, debido a diversas bondades que presenta sobre el tradicional an\'alisis de regresi\'on a la media. Adem\'as, a través del paradigma bayesiano permite incorporar conocimiento previo del fen\'omeno, y presenta una gran flexibilidad, al contar con componentes no param\'etricos. Asimismo, se abordan los modelos tradicionales de regresi\'on, para entender las ventajas que presenta el nuevo modelo.

El cap\'itulo 1 describe la importancia de las aproximaciones distintas a la regresi\'on a la media, así como la evoluci\'on hist\'orica de este tipo de modelos. El cap\'itulo 2 introduce al paradigma bayesiano y sus fundamentos generales. El cap\'itulo 3 se centra en los modelos bayesianos tradicionales de regresi\'on, tanto a la media, como sobre cuantiles. El cap\'itulo 4 plantea la especificaci\'on no param\'etrica particular del modelo de esta tesis, separ\'andolo de los tradicionales. El cap\'itulo 5 explica el algoritmo necesario para realizar inferencia y predicci\'on. El cap\'itulo 6 muestra simulaciones y aplicaciones del modelo, as\'i como los resultados obtenidos de evaluarlo en diversos conjuntos de datos. Finalmente, el cap\'itulo 7 hace referencia a las conclusiones de esta tesis, adem\'as de describir el trabajo futuro que se podr\'ia desarrollar.