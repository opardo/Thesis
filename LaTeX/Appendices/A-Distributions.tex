\chapter[Distribuciones de probabilidad]{Distribuciones de probabilidad }\label{chap:Distributions}

\section{Distribuci\'on Normal condicional}

\begin{prop*}
    Sea $X \in \mathbb{R}^m$ un vector aleatorio que tiene distribuci\'on Normal conjunta y est\'a particionado de la siguiente manera:

    \begin{equation*}
    \begin{aligned}
        X = 
        &\left[
        \begin{array}{c}
            X_1  \\
            X_2
        \end{array}
        \right], \\
        \text{ con dimensiones }
        &\left[
        \begin{array}{c}
            (m-q)  \\
            q
        \end{array}
        \right].
    \end{aligned}
    \end{equation*}
    
    Entonces, la media $\mu \in \mathbb{R}^m$ y varianza $\Sigma \in \mathbb{R}^{m \times m}$ de $X$ se pueden escribir
    \begin{equation*}
    \begin{aligned}
        \mu = 
        &\left[
        \begin{array}{c}
            \mu_1  \\
            \mu_2
        \end{array}
        \right], \\
        \text{ con dimensiones }
        &\left[
        \begin{array}{c}
            (m-q)  \\
            q
        \end{array}
        \right], y\\
        \Sigma = 
        &\left[
        \begin{array}{cc}
            \Sigma_{11} & \Sigma_{12}  \\
            \Sigma_{21} & \Sigma_{22}
        \end{array}
        \right], \\
        \text{ con dimensiones }
        &\left[
        \begin{array}{cc}
            (m-q) \times (m-q)  & (m-q) \times q  \\
            q \times (m-q) & q \times q
        \end{array}
        \right].
    \end{aligned}
    \end{equation*}
    
    La distribuci\'on condicional de $X_2$, sujeta a que $X_1 = a$ es Normal con $X_2|X_1=a \sim \mathcal{N}(X_2|\bar{\mu},\bar{\Sigma})$, donde
    
    \begin{equation*}
    \begin{aligned}
        \bar{\mu} &= \mu_2 + \Sigma_{2,1}\Sigma_{11}^{-1}(a-\mu_1) \\
        \bar{\Sigma} &= \Sigma_{22} - \Sigma_{21}\Sigma_{11}^{-1}\Sigma_{12}.
    \end{aligned}
    \end{equation*}
\end{prop*}

\section{Distribuci\'on de Dirichlet}

\begin{defin}
    Se dice que un vector aleatorio $x \in \mathbb{R}^n$ se distribuye de acuerdo a la \textbf{distribuci\'on de Dirichlet}  $\mathbf{(x \sim Dir(\alpha))}$ con vector de par\'ametros $\alpha$, espec\'ificamente,
    \begin{equation*}
        x = 
        \left(\begin{array}{c}
            x_1  \\
            \vdots \\
            x_n
        \end{array}\right),
        \qquad
        \alpha = 
        \left(\begin{array}{c}
            \alpha_1  \\
            \vdots \\
            \alpha_n
        \end{array}\right),
    \end{equation*}
    para los cuales se cumplen las restricciones
    \begin{equation*}
    \begin{aligned}
        x_i > 0, \forall i &\in \{1,...,n\} \\
        \sum_{i=1}^n x_i &= 1 \\
        \alpha_i > 0, \forall i &\in \{1,...,n\},
    \end{aligned}
    \end{equation*}
    si su funci\'on de densidad  es
    \begin{equation*}
        f(x|\alpha) = 
        \frac {1}{\mathrm {B} (\alpha)}
        \prod _{i=1}^{n}x_{i}^{\alpha _{i}-1},
    \end{equation*}
    donde $\mathrm{B}$ es la funci\'on Beta multivariada, y puede ser expresada en t\'erminos de la funci\'on $\Gamma$ como 
    \begin{equation*}
       \mathrm{B}(\alpha)=
       \frac {\prod _{i=1}^{n}\Gamma (\alpha _{i})}
       {\Gamma \left(\sum _{i=1}^{n}\alpha _{i}\right)},
       \qquad 
       \alpha =(\alpha _1,\cdots ,\alpha _n). 
    \end{equation*}
    
    La esperanza y varianza de cada $x_i$ son los siguientes:
    
    \begin{equation*}
    \begin{aligned}
        \mathbb{E}[x_i] &= \frac{\alpha_i}{\sum_{k=1}^n \alpha_k} \\
        Var(x_i) &= \frac
        {\alpha_i \left( \sum_{k=1}^n \alpha_k - \alpha_i \right)}
        {\left( \sum_{k=1}^n \alpha_k \right)^2 \left( \sum_{k=1}^n \alpha_k + 1 \right)}
    \end{aligned}
    \end{equation*}
    
\end{defin}

Es com\'un que esta distribuci\'on sea usada como la inicial conjugada de la distribuci\'on multinomial, debido a que el vector $x$ tiene las mismas propiedades de una distribuci\'on de probabilidad discreta (elementos positivos y que en conjunto suman 1).