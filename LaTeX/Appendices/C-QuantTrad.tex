\chapter{Predicci\'on de cuantiles, utilizando el modelo tradicional de regresi\'on a la media} \label{trad_mean_alg}

Una vez ajustado el modelo tradicional de regresi\'on a la media (descrito en la secci\'on \ref{trad_mean_reg}), un problema de inter\'es es obtener la distribuci\'on predictiva de alg\'un cuantil en espec\'ifico de la variable de respuesta, dado un nuevo valor en las covariables.

Recordando los supuestos del modelo, se tiene que
\begin{equation*}
    y = x^T\beta + \varepsilon,
\end{equation*}
con $\varepsilon \sim \mathcal{N}(0,\sigma^2)$, tal que $\beta$ y $\sigma^2$ se piensan como constantes, pero desconocidas. 

Dicha incertidumbre se traduce en una distribuci\'on de probabilidad para los par\'ametros, conocida como \textit{Normal-Gamma Inversa (NGI)}. Una de las mayores ventajas de usar tal distribuci\'on es que es conjugada respecto a la distribuci\'on Normal de los datos. Por lo tanto, una vez observados los datos, la distribuci\'on posterior de los par\'ametros $(\beta, \sigma^2)$ continuar\'a siendo \textit{NGI}.

Suponiendo que se tiene un nuevo vector de covariables $x_* \in \mathbb{R}^n$ para el cual se desea hacer una predicci\'on $y_*$ de la variable de respuesta, su distribuci\'on de probabilidad es
\begin{equation*}
    y_* | x_*, \beta, \sigma^2 \sim \mathcal{N}(x_*^T\beta, \sigma^2).
\end{equation*}

Si el inter\'es del modelador es estimar el cuantil \textit{p-\'esimo} de $y_*$, se tiene entonces que
\begin{equation*}
    q_p(y_*|x_*, \beta, \sigma^2) = x_*^T\beta + q_p(\varepsilon|\sigma^2),
\end{equation*}
con $q_p(\varepsilon|\sigma^2)$ el cuantil \textit{p-\'esimo} de $\mathcal{N}(0,\sigma^2)$.

Como es posible notar en el resultado anterior, una vez que se tiene una realizaci\'on de la distribuci\'on posterior de los par\'ametros $(\beta, \sigma^2)$, obtener una realizaci\'on de la predicci\'on del cuantil \'unicamente requiere de la aplicaci\'on de una funci\'on, dejando a un lado toda aleatoriedad. En ese orden de ideas, ser\'a posible estimar la distribuci\'on posterior predictiva del cuantil \textit{p-\'esimo} mediante la simulaci\'on de un n\'umero grande de sus propias realizaciones, a partir de simular valores de la distribuci\'on posterior de los par\'ametros. 

Sea $k$ el n\'umero de simulaciones a obtener de la distribuci\'on predictiva, y recordando que $(\beta,\sigma^2) \sim \mathcal{NGI}(\bar{M},\bar{V},\bar{a},\bar{b})$ (la distribuci\'on posterior de los par\'ametros), se puede reescribir como
\begin{equation*}
\begin{aligned}
    \sigma^2 &\sim \mathcal{GI}(\bar{a},\bar{b}), \\
    \beta &\sim \mathcal{N}(\bar{M},\sigma^2 \bar{V}).
\end{aligned}
\end{equation*}
Se tiene, entonces, que el algoritmo para obtener realizaciones de la distribuci\'on posterior predictiva del cuantil \textit{p-\'esimo} se puede representar de la siguiente forma.


\\ \\
\begin{algorithm}[H]
 {
     \For{$t=1,2,...,k$}
     {
        $1. \text{ Simular } \sigma^2_t \sim \mathcal{GI}(\bar{a},\bar{b})\;$\\
        $2. \text{ Simular } \beta_t \sim \mathcal{N}(\bar{M},\sigma^2_t \bar{V})\;$\\
        $3. \text{ Obtener } q_p(y_*)_t = x_*^T\beta_t + q_p(\varepsilon|\sigma^2_t) \;$
     }
 }
 \caption{Simulador de la distribuci\'on predictiva del cuantil \textit{p-\'esimo}, usando el modelo tradicional de regresi\'on a la media.}
\end{algorithm}
\BlankLine

\newpage